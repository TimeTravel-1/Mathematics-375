\documentclass[11pt]{article}

%%%%%%%%%%%%%% LATEX SAMPLE FILE %%%%%%%%%%%%%%%%
% A line which starts with a % sign
% is called a COMMENT. It is IGNORED
% by the LaTeX processor.

% Include math
\usepackage{amsmath,amsthm,amssymb}
% Include links
\usepackage{hyperref}


%%%%%%%%%%%%%  THEOREMS  %%%%%%%%%%%%%%%%%
% Let's define some theorem environments
% To use later in the paper
\theoremstyle{plain} % other options: definition, remark
\newtheorem*{theorem}{Theorem}
\newtheorem*{lemma}{Lemma}
% By including [theorem], the lemma follows the numbering of theorem
% e.g. Thm 1, Lemma 2, Thm 3, Thm 4, \dots
\theoremstyle{definition}
\newtheorem*{definition}{Definition} % the star prevents numbering

\theoremstyle{example}
\newtheorem*{example}{Example}
% Remarks
\theoremstyle{remark}
\newtheorem*{remark}{Remark}




%%%%%%%%%%%%%%  PAGE SETUP %%%%%%%%%%%%%%%%%
% LaTeX has big default margins
% The following sets them to 1in
\usepackage[margin=1.5in]{geometry}

% The following sets up some headers
\usepackage{fancyhdr}
\pagestyle{fancy}
\lhead{Differential Equations} % Left Header
\rhead{\thepage} % Right Header
\cfoot{} % Center Foot (empty)






%%%%%%%%%%%%% SHORTCUTS %%%%%%%%%%%%%%%%%%%%
% You can define your own shortcuts too.
% Examples of custom commands
\newcommand{\half}{\frac{1}{2}}
\newcommand{\cbrt}[1]{\sqrt[3]{#1}}

\begin{document}

% Set up a title
\title{Math 375}
\author{David Ng}
\date{Fall 2016}
\maketitle

% This line makes a ToC
\tableofcontents

% This line starts a new page
\eject

%%%%%%%%%%%%% January 11 %%%%%%%%%%%%%%%%%%%%

\section{September 12, 2016}
\subsection{Review of Differential Equations}
 \begin{definition}
 A \textbf{differential equation} is an equation connecting an unknown function with some of its derivatives.  A function that turns an equation into an identity is a solution of the given differential equation. A differential equation includes dependent and independent variables. 

 \end{definition}
 
 
 \begin{example}
 Suppose we are given the equation $y=f(x)$. What are the dependent and independent variables?
 \end{example}

 In the above equation, $x$ is independent while $y$ is dependent. 
 
 \begin{example}
Given that $N=N(t)$ is the population size, and $r$ is a positive constant, find the solution to the differential equation $$\frac{\mathrm d N}{\mathrm d t} = rN$$ 
 \end{example}

We note that the function $N(t) = Ce^{rt}$ is a solution, since

$$\frac{\mathrm d N}{\mathrm d t} = Cre^{rt} = r\left(Ce^{rt}\right) = rN$$

\begin{example}
Solve the differential equation $\frac{\mathrm d^2y}{\mathrm d t^2} = -g$, where $g$ is the gravitational constant.
\end{example}

To solve this, we need to determine the solution to $y(t)$. Let us integrate to arrive at the solution.

\begin{align*}
	y'(t) &= v(t)  \\
		&= \int y''(t)\mathrm d t\\
		&= \int (-g)\mathrm d t\\
		&= -gt+C_1
\end{align*}

\begin{align*}
	y(t) &= \int y'(t) \mathrm d t\\
	&= \int v(t)\\
	&= \int \left(-gt+C_1\right)\mathrm d t\\
	&= -\frac{gt^2}{2}+C_1t+C_2
\end{align*}

Generally, a differential equation has an infinite number of solutions. The process of determining the solution is referred to as integration. The number of constants is equal to the order of the equation (which is the highest order of the derivative involved. A collection of solutions is called the general solution. In order to find a particular solution, we have to introduce initial conditions.

\begin{example}
Given the initial conditions $y(0) = 15$m, and $v(0) = 0.5$m/s, solve the differential equation $\frac{\mathrm d^2y}{\mathrm d t^2} = -g$, where $g$ is the gravitational constant.
\end{example}

We note that we solve for the constant values given the initial conditions. We get that $C_1 = 0.5$ and $C_2 = 1.5$. Thus, the particular solution is $$y = -\frac{gt^2}{2}+\frac{t}{2}+1.5$$

\begin{example}
Classify the following equations $$e^s \frac{\mathrm d^3 u}{\mathrm d s^3} = \sin(s^5)\frac{\mathrm d u}{\mathrm d s} - s^5\ln(s^4+s^2+1)$$
$$\left(\frac{\mathrm d u}{\mathrm d s}\right)^2+5u = s+7$$
\end{example}

The first equation is a third-order linear equation, while the second equation is nonlinear.

\section{September 14, 2016}
\subsection{General Differential Equations Cont'd}

An ordinary $n$th order differential equation is of the form $$y^{(n)}= f\left (x, y, y',...,y^{(n-1)}\right)$$

We introduce initial conditions as follows,

$$y(x_0) = y_0, y'(x_0)=y'_0,...,y^{(n-1)}(x_0) = y_0^{(n-1)}$$

By introducing initial conditions, we get the initial value problem. 

\subsection{Linear and Non-Linear Differential Equations}


\begin{definition}
A linear differential equation is of the form $$y^{(n)} = a_{n-1}(x)y^{(n-1)} + ... + a_0(x)y + B(x)$$
\end{definition}

\begin{example}
Classify the equation $\frac{\partial u}{\partial t} = k\frac{\partial^2 u}{\partial x^2}$.
\end{example}

We note that this is the heat transfer equation with $u(t,x)$. This is a linear partial differential equation. 

\begin{remark}
First order differential equations can also be written in a differential form.
\end{remark}
\begin{example}
Newton's law of cooling/heating states that $\frac{\mathrm d T}{\mathrm d t} = -k(T-T_{medium})$, where $T(t)$ is the temperature. We note that the right hand side does not depend on $t$. Such equations of the form $y' = g(y)$ are called \textbf{autonomous}. Classify this differential equation.
\end{example}

We note that we can rewrite the equation as $$\mathrm d T = -k(T-T_{medium})\mathrm d t$$ This is a first order ordinary differential equation.

\subsection{First Order Linear Equations}

First order linear differential equations are of the form,

$$y'(t) = p(t)y(t)= q(t)$$

\begin{remark}
We note that for $(\mu(t)y(t))' = \mu y' + \mu' y$, we denote $\mu(t)=e^{\int p(t)\mathrm d t}$ as the \textbf{integrating factor}
\end{remark}


We recall that \begin{align*}\frac{\mathrm d}{\mathrm d t}\left(e^{\int p(t)\mathrm d t}\right) &= e^{\int p(t)\mathrm d t} \frac{\mathrm d }{\mathrm d t}\left(\int p(t)\mathrm d t\right)\\
&= p(t)e^{\int p(t)\mathrm d t}
\end{align*}

Here, we multiply the equation $y'(t) = p(t)y(t)= q(t)$ by $\mu(t)$. Thus, we get 

\begin{align*}
	e^{\int p(t) \mathrm d t}y'(t) + p(t)e^{\int p(t) \mathrm d t}y(t) &= q(t) e^{\int p(t) \mathrm d t}\\
	\mu y'(t) + \mu(t)y(t) &= q(t)e^{\int p(t) \mathrm d t}\\
	&= \frac{\mathrm d}{\mathrm d t} \left(e^{\int p(t) \mathrm d (t)}y(t) \right)
\end{align*}

However, we note that $e^{p(t) \mathrm d t} y(t) = \int q(t) e^{\int p(t) \mathrm d t} + C$. Thus, 
$$y(t) = Ce^{-\int p(t) \mathrm d t} + e^{-\int p(t) \mathrm d t}\int q(t)e^{\int p(t) \mathrm d t}\mathrm d t$$


\begin{example}
	Solve $$xy' + 3y-x^2=0$$
\end{example}

First, we rewrite the equation in standard form. That is, the equation becomes $y' + \frac{3}{x}y = x$. We now identify $p(t)$ from the general form, which in this case is $\left(\frac{3}{x}\right)$. Thus, we note that the integrating factor is 

\begin{align*}
\mu(x) &= e^{\int \frac{3}{x} \mathrm d x} \\
&= e^{3 \ln(x)} \\
&= \left(e^{\ln(x)}\right)^3 \\
&= x^3
\end{align*}

Now, we multiply the original equation $y'+\frac{3}{x}y=x$ by $x^3$ to get the equation 

\begin{align*}
x^3y' + 3x^2y&=x^4\\
\frac{\mathrm d}{\mathrm d x}x^3y &= x^4\\
x^3y &= \int x^4\mathrm d x \\
&= \frac{x^5}{5}+C
\end{align*}

Thus, we now isolate $y$ to get it in its explicit form $$y = \frac{x^2}{5}+\frac{C}{x^3}$$


\section{September 16, 2016}
\subsection{First Order Linear Equations Cont'd}


\begin{example}
Solve the initial value problem given that $y(0) = 4$ for the following equation,

$$\cos(t)y'=y\sin(t)+\cos^2(t)$$
\end{example}

To solve this equation, we rewrite the equation as $y' + p(t)y = q(t)$. The equation becomes $y' -\tan(t)y = \cos(t)$. We now find the integrating factor, which is $e^{\int p(t)\mathrm d t}$, 

\begin{align*}
	e^{-\int\tan(t) \mathrm d t} &= e^{\ln(\cos(t))}\\
	&= \cos(t)
	\end{align*}
	
	We now multiply the equation $y' - \tan(t)y = \cos(t)$ to get $\cos(t)y' - \sin(t)y = \cos^2(t)$. We note the left hand side of the equation becomes $\left(\cos(t)y\right)' = \cos^2(t)$. Thus by taking the integral of both sides, $\cos(t)y = \int\cos^2(t) \mathrm d t$. 
	
	\begin{align*}
		\cos(t)y &= \int\cos^2(t) \mathrm d t\\
		&= \int \frac{1+\cos(2t)}{2} \mathrm d t \\
		&= \frac{1}{2}t + \frac{\sin(2t)}{4}+ C \\
		y &= \frac{t}{2\cos(t)} + \frac{2\sin(t)\cos(t)}{4\cos(t)} + \frac{C}{\cos(t)}\\
		&= \frac{t}{2\cos(t)}+ \frac{\sin(t)}{2} + \frac{C}{\cos(t)}
	\end{align*}
	
With $y(0)=4$, we find by substituting into the equation that $C = 4$. Thus, 

$$y = \frac{t}{2\cos(t)} + \frac{\sin(t)}{2} + \frac{4}{\cos(t)}$$

\begin{theorem}
If the coefficients $p$ and $q$ of the equation $y' + p(t)y = q(t)$ are defined and continuous on an interval $(a,b)$including $t_0$, then the solution of $y' + p(t)y = q(t)$ with the initial condition $y(t_0) = y_0$ exists on $(a,b)$ and is unique. 
\end{theorem}

\begin{example}
Find the interval on which the solution of the equation $\left(t^2-9\right)y' + \ln(t+7)y = \cos(3t)$ is guaranteed to have a unique solution if $y(0)=5$ and $y(-5) = 2$. 
\end{example}

We first note that $p(t) = \frac{\ln(t+7)}{t^2-9}$ and $q(t) = \frac{\cos(3t)}{t^2-9}$. From $t^2-9$, we note that $t \neq\pm 3$. Additionally, $\ln(t+7)$ means that $t+7 > 0$, so $t > -7$. Therefore, the domain is $\{t|-7 < t < -3, -3 < t < 3, 3<t\}$.

\begin{definition}
An equation of the form $\frac{\mathrm d y}{\mathrm d x} = f(x)g(y)$ is \textbf{separable} since we can separate $x$ and $y$. 
\end{definition}To solve the equation, we would rewrite the equation as $\frac{\mathrm d y}{g(y)} = f(x) \mathrm d x$. Then integrate both sides. 


\begin{example}
Given that $y(1) = 1$ or $y(1) = -2$, solve $$y' = -\frac{x}{y}$$
\end{example}

We note that we can rewrite as $\frac{\mathrm d y}{\mathrm d x} = -\frac{x}{y}$. That is, $y\mathrm d y = -x\mathrm d x$. Integrating both sides, we get 

\begin{align*}
\int y\mathrm d y &= \int x \mathrm d x\\
\frac{y^2}{2} &= -\frac{x^2}{2} + C_1 \\
y^2 &= 2C_1-x^2 \\
y &= \pm \sqrt{C-x^2}
\end{align*}

Given that $y(1)=1$, we note that $C = 2$, so $y = \sqrt{2-x^2}$ exists on $(-\sqrt{2}, \sqrt{2})$. For $y(1) = -2$, we find that $C = 5$. Thus, $y=-\sqrt{5-x^2}$ exists on the interval $(-\sqrt{5}, \sqrt{5})$.

\section{September 19, 2016}
\subsection{First Order Separable Equations}

There may be situations where it is difficult to express $y=y(x)$. 

\begin{remark}
We note that if we simplify $x^2=x$ to $x=1$, we have lost a solution at $x=0$. 
\end{remark}

\begin{example}
Find the general solution of 

$$\frac{\mathrm d y}{\mathrm d x}\left(1+x^2\right) = \frac{1}{2}\left(y^2-1\right)$$
\end{example}

We separate $x$ and $y$. The equation becomes$$\frac{\mathrm d y}{y^2-1} = \frac{\mathrm d x}{1+x^2}$$ We note that we divided by $1+x^2$ and $y^2-1$. We find that $y=\pm 1$ is a solution. We now take the integral of both sides to get

$$ \int \frac{\mathrm d y}{y^2-1} =\int  \frac{\mathrm d x}{1+x^2}$$

We employ partial fraction to get 

$$\frac{2}{(y-1)(y+1)} = \frac{A}{y-1} + \frac{B}{y+1}$$

Solving for $A$ and $B$ using $y = \pm 1$, we get $A = 1$ and $B = -1$. Thus, the integral becomes

\begin{align*}
	\int \frac{\mathrm d y}{y-1} - \int \frac{\mathrm d y}{y+1} &= \int \frac{\mathrm d x}{1+x^2} \\
	\ln(y-1) - \ln(y+1) &= \arctan(x)+C_1\\
	\frac{\ln(y-1)}{\ln(y+1)} &= C_1 + \arctan(x)\\
	\frac{|y-1|}{|y+1|} &= e^{C_1}e^{\arctan(x)}\\
	\frac{y-1}{y+1} &= \pm C_2e^{\arctan(x)}\\
\end{align*}
We note that for $C=0$, $y=1$.
\begin{align*}
	\frac{y-1}{y+1} &= Ce^{\arctan(x)}\\
	y-1 &= yCe^{\arctan(x)} + Ce^{\arctan(x)}\\
	y\left(1-Ce^{\arctan(x)}\right) &= 1+Ce^{\arctan(x)}\\
	y &= \frac{1+Ce^{\arctan(x)}}{1-Ce^{\arctan(x)}}
\end{align*}

We note that this is the general solution, whereas $y=-1$ is a \textbf{singular solution}.

\subsection{Types of Equations}
\begin{enumerate}
	\item Linear equations are of the form $\left(\mu(t)y(t)\right)' = f(t)$
	\item Separable equations are of the form $y'(t) = f(t)g(y)$
\end{enumerate}

Let us suppose that we have a solution $F(x,y) = C$ and a differential equation $\frac{\mathrm d}{\mathrm d x}F(x,y) = 0$. 

\subsection{First Order Exact Equations}

\begin{definition}
An equation of the form $\mu(x,y)\mathrm d x + N(x,y)\mathrm d y = 0$ is \textbf{exact} if $\mu_y = N_x$. We find the potential function $F(x,y)$ such that $\mu=F_x = \frac{\partial F}{\partial x}$, and $N = F_y$. The general solution if $F(x,y) = C$. 
\end{definition}

\begin{example}
Find the general solution of $$\left(2xy + 3y^2\right)\frac{\mathrm d y}{\mathrm d x} + y^2 + \cos(x) = 0$$
\end{example}

We write the equation in the differential form. That is

$$\left(y^2 + \cos(x)\right) \mathrm d x + \left(2xy + 3y^2\right) \mathrm d y = 0$$

We now determine if this is exact. $\mu_y = \frac{\mathrm d}{\mathrm d y}\left(y^2 \cos(x)\right) = 2y$. $N_x = \frac{\mathrm d }{\mathrm d x}\left(2xy + 2y^2\right) = 2y$. We note that $2y = 2y$. Hence, it is exact. 

\begin{align*}
\frac{\partial F}{\partial x} &= M \\
	&= y^2+\cos(x)\\
	F(x,y) &= \int \left(y^2 + \cos(x)\right) \mathrm d x\\
	&= y^2x + \sin(x) + k(y)
\end{align*}

\begin{align*}
\frac{\partial F}{\partial y} &= 2yx+ 0 + k'(y)\\
	&= N \\
	&= 2xy + 3y^2\\
\end{align*}

Since $k'(y) = 3y^2$, we get that $k(y) = y^3$. Thus, $F(x,y) = y^2x + \sin(x) + y^3$. The general solution is therefore

$$xy^2 + \sin(x) + y^3  = C$$

\begin{example}
Solve $$\left(y\cos(x) + 2xe^y\right)\mathrm d x + \left(\sin(x) + x^2e^y-1\right) \mathrm d y = 0$$
\end{example}

Firs, we note that this is an exact equation.

\begin{enumerate}
	\item $M_y = \cos(x) + 2xe^y$
	\item $N_x = \cos(x) + 2xe^y$
\end{enumerate}

Now, we have $F = \int M \mathrm d x = y\sin(x) + x^2e^y + k(y)$. $F_y = \sin(x) + x^2e^y + k'(y)$, where $k'(y) = -1$. Thus, $k(y) = -t$. $F(x,y) = y\sin(x) + x^2e^y-y$. The general solution is therefore

$$y\sin(x) + x^2e^y-y=C$$

\section{September 21, 2016}

\subsection{Types of First Order Differential Equations}

\begin{example}
For which $a$ and $b$ will $$ax^3y^2 + 6x^byy' = 0$$ be exact?
\end{example}

We can rewrite $y' $ as $\frac{\mathrm d y}{\mathrm d x}$. We can then rewrite the equation as $$ax^3y^2 \mathrm d x + 6x^by\mathrm d y = 0$$ where $M$ is the term before $\mathrm d x$ and $N$ is the term before $\mathrm d y$. Thus, $M_y= 2ax^3y$, and $N_x = 6bx^{b-1}y$. Since we know that $M_y=N_x$, then we get $a=12$ and $b=4$. 

\subsection{Bernoulli Equations}

\begin{definition}A \textbf{Bernoulli equation} is of the form
$$\frac{\mathrm d y}{\mathrm d t} + p(t) y = q(t)y^n$$
where $n \neq 0, 1$. 
\end{definition}

We note that we can manipulate this equation by dividing by $y^n$ to get

$$\frac{y'}{y^n} = p(t)y^{1-n} = q(t)$$
We denote $z=y^{1-n}$ to get

$$\frac{\mathrm d z}{\mathrm d t} = (1-n)\left(y^{-n}y'\right)$$ The equation therefore becomes
$$\frac{1}{1-n}z' + pz = q$$

\begin{example}
Solve $$y' + y + xy^2 = 0$$
\end{example}

We first divide by $y^2$ to get the equation in the appropriate form. The result is $\frac{y'}{y^2}+ \frac{1}{y} = -x$. In this case, $z = \frac{1}{y}$. Thus, $z' = -\frac{1}{y^2}y'$. By substituting into the equation, we get $$z' - z = x$$

We now multiply by the integrating factor, $e^{-\int \mathrm d x} = e^{-x}$. The integral becomes the following, to which we apply integration by parts. 

\begin{align*}
e^{-x}z &= \int xe^{-x}\mathrm d x \\
&= -xe^{-x}+\int e^{-x} \mathrm d x\\
&= -xe^{-x}-e^{-x} + C\\
e^xe^{-x}z &= e^x\left(-xe^{-x}-e^{-x} + C\right)\\
z &= Ce^x-x-1
\end{align*}

Since $z = \frac{1}{y}$, substituting back, we get $$y = \left(Ce^x -x-1\right)^{-1}$$

\begin{example}
Solve $$ty' + y = \frac{10t^2+3}{y^2}$$
\end{example}

We multiply both sides by $y^2$ to get $$ty'y^2 + y^3 = 10t^2+3$$ We let $z=y^2$, and thus $z' = 3y^2y'$. The equation becomes

$$\frac{1}{3}tz' + z = 10t^2 + 3$$

At this point, we solve through use of integrating factor. That is, the equation becomes the following after determining the integrating factor $e^{\int \frac{3}{t} \mathrm d t} = t^3$

\begin{align*}
z' + \frac{3}{t}z &= 30t + \frac{9}{t} \\
t^3z' + 3t^2z &= 30t^4 + 9t^2\\
t^3z &= \int\left(30t^4 + 9t^2\right)\mathrm d t\\
&= 6t^5+3t^3+C\\
z &= 6t^2 + 3 + \frac{C}{t^3}\\
\end{align*}

We now substitute $y$ back into the equation to get $$y = \left(3 + 6t^2 + \frac{C}{t^3}\right)^{\frac{1}{3}}$$





\subsection{Homogeneous Equations}

\begin{definition}
A \textbf{homogeneous equation} is of the form $$\frac{\mathrm d y}{\mathrm d x} = f\left(\frac{y}{x}\right)$$
\end{definition}

\begin{remark}
If all the terms have the same degree, then the following ratio of two polynomials holds$$\frac{p(x,y)}{q(x,y)} = f\left(\frac{y}{x}\right)$$ \end{remark}

To solve, we denote $u = \frac{y}{x}$, so that $y = ux$ and $y' = u'x+u$. We note that $y'x+u = f(u)$. Therefore, we have the following separable equation

\begin{align*}
x\frac{\mathrm d u }{\mathrm d x} &= f(u) - u\\
\int \frac{\mathrm d u}{f(u)-u} &= \int \frac{\mathrm d x}{x}
\end{align*}


\begin{example}
Solve $$y' = \frac{y}{x} + \tan\left(\frac{y}{x}\right)$$
\end{example}

We denote $u = \frac{y}{x}$. Doing so, we get that $y = xu$ and $y' = u + xu'$. Thus, we get the equation $u + xu' = u + \tan(u)$. Separating this equation, we get









\begin{align*}
x\frac{\mathrm d u}{\mathrm d x} &= \tan(u) \\
&= \frac{\sin(u)}{\cos(u)} \\
\int \frac{\cos(u)}{\sin(u)}\mathrm d u &= \int \frac{\mathrm d x}{x} \\
&= \ln(x) + C_2\\
\ln(\sin(u)) &= \ln(x)+\ln(C_3)\\
&= \ln(C_3x)\\
\end{align*}
At this point, we note that $C_3 > 0$ or $C_3 < 0$. Also, $C_3$ is included as well. Thus, we have $\sin(u) = \pm C_3x$. Substituting, we get that $\sin\left(\frac{y}{x}\right) = \pm C_3x$. 



\section{September 23, 2016}
\subsection{Homogeneous Equations Cont'd}

\begin{example}
Given that $y(1)=0$, solve the following equation

$$(x-y)y' + x+y=0$$
\end{example}

We note that by rearranging the equation, we get 

\begin{align*}
	y'&=-\frac{x+y}{x-y}\\
	&= \frac{x+y}{y-x}\\
	&= \frac{1+\frac{y}{x}}{\frac{y}{x}-1} \\
	&= f\left(\frac{y}{x}\right)
	\end{align*}

We now let $u =\frac{y}{x}$. Thus, $y=xu$ and $y' = u+xu'$. Substituting into the equation, we get

\begin{align*}
	u + xu' &= \frac{1+u}{u-1} \\
	xu' &= \frac{1+u}{u-1}-u\\
	x\frac{\mathrm d u}{\mathrm d x} &= \frac{-u^2+2u+1}{u-1}\\
\end{align*}

Taking the integral of both sides, we get

\begin{align*}
	\int\frac{(u-1)\mathrm d u}{1+2u-u^2} &= \int \frac{\mathrm d x}{x}\\
\end{align*}

We note here that the derivative of $1+2u + u^2$ is $2-2u$. We make the substitution to let $w = 1+2u-u^2$, and $\mathrm d w = -2(u-1)\mathrm d u$. Solving this, we get

\begin{align*}
	\ln\left(1+2u-u^2\right) &= \ln\left(\frac{C}{x^2}\right)\\
	1+2u-u^2 &= \frac{C}{x^2}\\
\end{align*}

By substituting $u = \frac{y}{x}$, we find that the general solution is 

$$x^2+2xy-y^2=C$$. We note that $y(1)=0$, so $$x^2+2xy-y^2=1$$ is a particular solution.

\subsection{Integrating Factor}
If $M(x,y)\mathrm d x + N(x,y) \mathrm d y = 0$ is not exact, we can sometimes find $M(x,y)$ such that $\mu M\mathrm d x + \mu N \mathrm d y$ is exact. When do we have $\mu(x)$? But we know that $\mu(x) M\mathrm d x + \mu (x) N \mathrm d y=0$ is exact, so

\begin{align*}
	\frac{\partial}{\partial y}(\mu M) &= \mu M_y\\
	&= \frac{\partial}{\partial x} \left(\mu (x)N\right)\\
	&= \mu ' N + \mu N_x \\
	&= \mu 'N \\
\end{align*}

Thus, we have 














$$\mu 'N = \mu(M_y-N_x)$$

or

$$\frac{\mu '}{\mu} = \frac{M_y-N_x}{N}$$

If $\frac{M_y-N_x}{N}$ does not depend on $y$, we have an integrating factor, since

\begin{align*}
	\left(\ln(\mu)\right)' &= \frac{M_y-N_x}{N}\\
	\ln(\mu) &= \int \frac{M_y-N_x}{N}\\
	\mu(x) &= e^{\frac{M_y-N_x}{N} \mathrm d x}
\end{align*}

Similarly, we have $\mu = \mu (y)$ if $\frac{N_x-M_y}{M}$ does not depend on $x$. 



\begin{example}
Solve $$\left(x+y^2\right)\mathrm d x -2xy\mathrm d y = 0$$i
\end{example}

We try the integrating factor $\mu(x)$, to get 

\begin{align*}
	\frac{M_y-N_x}{N} &= \frac{2y-(-2y)}{-2xy}\\
	&= \frac{-4y}{2xy} \\
	&= -\frac{2}{x}\\
	\frac{\mu '}{\mu} &= -\frac{2}{x}\\
	(\ln(\mu))' &= -\frac{2}{x}\\
	\ln(\mu) &= \int \left(-\frac{2}{x}\right)\mathrm d x \\
	&= -2\ln(x)+C\\
	&= -\ln(x^2) \\
	&= \ln\left(\frac{1}{x^2}\right)
	\end{align*}
	
	Thus, we get that $\mu(x) = \frac{1}{x^2}$. Now, by applying this integrating factor, we solve the original equation.
	
\begin{align*}
	\frac{x+y^2}{x^2}\mathrm d x - \frac{2xy}{x^2}\mathrm d y &= 0 \\
	\left(\frac{1}{x} + \frac{y^2}{x^2}\right) \mathrm d x - 2\left(\frac{y}{x}\right) \mathrm d y &= 0\\
\end{align*}

Thus, we note that 
$$F(x,y) = \int \left(\frac{1}{x} + \frac{y^2}{x^2}\right) \mathrm d x = \ln(x) -\frac{y^2}{x} + k(y)$$

By finding $\frac{\partial F}{\partial y} = -2\frac{y}{x}+k'(y)$, we note that $k'(y) =0$. Thus, the solution is 

$$F(x,y) = \ln(x) -\frac{y^2}{x} = C$$

\begin{example}
Solve $$y(1+xy)\mathrm d x - x \mathrm d y = 0$$
\end{example}

We note that in this equation, we designate $M = y(1+xy)$ and $N = -x$. Thus, $N_x = -1$ and $M_y = 1+2xy$. Therefore, 

\begin{align*}
\frac{N_x-M_y}{M} &= \frac{-1-1-2xy}{y(1+xy)} \\
&= -\frac{2(1+xy)}{y(1+xy)} \\
&= -\frac{2}{y}\\
\frac{\mu '(y)}{\mu} &= -\frac{2}{y}\\
\ln(\mu) &= -2\ln(y)\\
\end{align*}

Thus, $\mu(y) = \frac{1}{y^2}$. Using this as the integrating factor, we find that $$F(x,y) = \frac{x}{y} + \frac{x^2}{2} = C$$

\begin{example}
Find $n$ and $m$ such that $\mu(x,y) = x^ny^m$ is an integrating factor and solve $$(6y+14x)\mathrm d x + \left(4x+6x^2y^{-1}\right)\mathrm d y = 0$$
\end{example}

We first multiply the entire equation by $x^ny^m$ to get 


$$\left(6x^ny^{m-1} + 14x^{n+1}y^m\right)\mathrm d x + \left(4x^{n+1}y^m + 6x^{n+2}y^{m-1}\right)\mathrm d y = 0$$, where the term before $\mathrm d x$ is $M$ and the term before $\mathrm d y$ is $N$. Finding $\frac{\partial M}{\partial y}$ and $\frac{\partial N}{\partial x}$, we then solve the system of equations. Doing this, we find that $n=5$ and $m=3$. Thus

$$F(x,y) = x^6y^4+2x^7y^3 = C$$




\section{September 26, 2016}
\subsection{Exponential Growth and Decay}

The equation that represents exponential growth and decay is 
$$\frac{\mathrm d y}{\mathrm d t} = ky$$ For $k>0$, this represents such relations as population growth or bank interest. For $k < 0$, this represents such relations as radioactive decay. We note that if we rearrange the equation and take the integral of both sides, we get $$\ln(y) = kt + C$$ Solving for $y$ while noting that $y(t_0) =y_0$, we get 
$$y = y_0e^{k(t-t_0)}$$.




\begin{example}
If a capital is doubled in 8 years and the interest $k$ is constant ($N' = kN, N > 0$), how long will it take for the capital to triple. 
\end{example}

We know that in 8 years, the capital doubles. Therefore, this is represented by the fact that 

$$ 2 = e^{8k}$$

Therefore, we note that $8k = \ln(2)$. Thus, $k = \ln(2)/8$. Now, we solve for the amount of time needed for the capital to double. We note that for the capital to triple, we need $y = 3y_0$. Combined with out knowledge of $k$, we get

$$3 = e^{\frac{\ln(2)t}{8}}$$

Thus, $\ln(3) = \ln(2)t/8$. Solving for $t$, we get

$$t= \frac{8 \ln(3)}{\ln(2)}$$

\subsection{Newton's Law of Cooling/Heating}

Let $T(t)$ be the temperature of the object, $S(t)$ be the temperature of the medium, and $k$ be the coefficient. The equation that represents Newton's Law of Cooling/Heating is 

$$\frac{\mathrm d T}{\mathrm d t} = k(S(t) -T(t))$$

We can simplify by taking the integrating factor to be $\mu (t) = e^{kt}$, 

\begin{align*}
	\frac{\mathrm d T}{\mathrm d t} +kT &= kS\\
	e^{kt}\frac{\mathrm d T}{\mathrm d t} +ke^{kt}T &= kSe^{kt}\\
	\left(e^{kt}T\right)' &= kS(t)e^{kt}\\
	e^{kt}T &= \int kS(t)e^{kt} \mathrm d t\\
	T &= Ce^{-kt}+e^{-kt}\int kS(t)e^{kt}\mathrm d t
\end{align*}



\begin{example}
A metal object is heated to $200^{\circ}\mathrm C$ degrees Celsius and then placed in a large room with a constant temperature of $20$ degree Celsius to cool. After 10 minutes, the temperature of the object is 100 degrees Celsius. When was it at 140 degrees Celsius. How long will it take to cool to 25 degrees Celsius. 
\end{example}

We note that $T(0)=200^{\circ}\mathrm C$, and then apply Newton's law to get

$$\frac{\mathrm d T}{\mathrm d t} = k(20-T)$$ 

This becomes 

\begin{align*}
e^{kt}T &= \int 20ke^{kt} \mathrm d t\\
	&= 20e^{kt} + C \\
	T(t) &= \left(20e^{kt}+C\right)e^{-kt}\\
	&= 20+Ce^{-kt} \\
\end{align*}

Since we know that $T(0) = 200$, we determine that $C = 180$. Now, we note that at $t=10$, the temperature was $100$. Thus, we solve for $k$ to obtain

\begin{align*}
T(t) &= 20+180e^{-10k}\\
T(10) &= 20 + 180e^{-10k}\\
100 &= 20+180e^{-10k}\\
80 &= 180e^{-10k}\\
\frac{4}{9} &= e^{-10k}\\
\ln\left(\frac{4}{9}\right) &= -10k\\
k &= -\frac{1}{10}\ln\left(\frac{4}{9}\right)
\end{align*}

Now to solve the first question, we note that we need to solve for when $T(t) = 140$. Thus, substituting into the equation, we get 

\begin{align*}
	20+180e^{-kt} &= 140\\
	e^{\frac{1}{10}\ln\left(\frac{4}{9}\right)t} &= \frac{2}{3}\\
	\frac{1}{10}\ln\left(\frac{4}{9}\right) t&= \ln\left(\frac{2}{3}\right)\\
	\frac{1}{10}\ln\left(\frac{2^2}{3^2}\right)t &=  \ln\left(\frac{2}{3}\right)\\
	\frac{2}{10}\ln\left(\frac{2}{3}\right) t &=  \ln\left(\frac{2}{3}\right)\\
	\frac{1}{5}t &=1\\
	t &= 5
\end{align*}

Thus, after 5 minutes, the metal object is at 140 degrees. To solve the second problem, we note that  we have $T(t) = 25$. By applying Newton's law, we get

\begin{align*}
20 + 180e^{-kt} &= 25\\
e^{-kt} &= \frac{5}{180}\\
-kt &= \ln\left(\frac{1}{36}\right)\\
\frac{1}{10}\ln\left(\frac{4}{9}\right)t &= \ln\left(\frac{1}{36}\right)\\
t &= \frac{-\ln(36)}{\frac{1}{10}\ln\left(\frac{4}{9}\right)}\\
&= \frac{10\ln(36)}{\ln(9)-\ln(4)}
\end{align*}

Thus, it will take around 44186 minutes for the metal object to cool to $25^{\circ}\mathrm C$. 






\section{September 28, 2016}
\subsection{Mixing Problems}


Let $Q(t)$ be the amount of salt at time $t$, $r_1$ be the incoming rate, $r_2$ be the outgoing rate, $C_1$ be the incoming concentration, and $C_2$ be the outgoing concentration. Thus, we note the relation 

$$\frac{\mathrm d Q}{\mathrm d t} = C_1r_1 - C_2r_2$$

We note that $C_2 = \frac{Q}{V}$, where $V = V_0 + (r_1-r_2)t$. Therefore, the above equation becomes

$$\frac{\mathrm d Q}{\mathrm d t} = C_1r_1 - \frac{Qr_2}{V_0+(r_1-r_2)t}$$

\begin{example}
A tank contains $1000L$ of water with $2kg$ of salt. A valve is opened so that water containing $0.02kg$ of salt per liter flows into the tank at a rate of $5L/min$. The mixture is well stirred and drains from the tank at a rate of $5L/min$. Find $Q(t)$, which is the amount of salt after $t$ minutes. Determine $Q_c = \lim_{t \rightarrow \infty}Q(t)$. When will $Q(t)$ be $99\%$ of $Q_c$.
\end{example}

We note that $r_1 = 5L/min$ and $r_2 = 5L/min$. We also note that $C_1 = 0.02kg/L$ and $C_2 = \frac{Q}{V}= \frac{Q}{1000}$. Thus we get 


\begin{align*}
	\frac{\mathrm d Q}{\mathrm d t} &= C_1r_1 - c_2r_2\\
	&= 0.02*5-\frac{5Q}{1000}\\
	Q' + \frac{Q}{200} &= 0.1\\
\end{align*}

We now find the integrating factor and find that it is $e^{\frac{t}{200}}$ to get

\begin{align*}
	Q' e^{\frac{t}{200}}+ \frac{Q}{200} e^{\frac{t}{200}}&= 0.1e^{\frac{t}{200}}\\
	\left(Q(t) e^{\frac{t}{200}}\right)'&= 0.1e^{\frac{t}{200}}\\
	Q(t)e^{\frac{t}{200}} &= \int 0.1e^{\frac{t}{200}} \mathrm d t\\
	&= 20e^{\frac{t}{200}} + C\\
	Q(t) &= 20+Ce^{-\frac{t}{200}}
\end{align*}

After solving this and noting that $Q(0) = 2$, we find that $C = -18$. Therefore

$$Q(t) = 20-18e^{-\frac{t}{200}}$$

To solve the second question, we note that as $t \rightarrow \infty$, the expression $Q(t)$ approaches 20.

To solve the third question, we note that we solve for when $Q(t) = .99*20 = 19.8$. We find that $t= 200\ln(90)$ minutes, which is approximately $15$ hours. 

\subsection{Electric Circuits}
We note the following laws

\begin{enumerate}

\item \textbf{Ohm's Law} states that $$E_R = V_R = RI = R\frac{\mathrm d Q}{\mathrm d t}$$


\item $$V_H = H\frac{\mathrm d I}{\mathrm d t} = E_H$$
\item $$E_c = V_c = \frac{Q}{c}$$

\item \textbf{Kirchoff's Law} states that 

$$E = E_R + E_L + E_c$$ 
\item LR	$$ E=RI + H\frac{\mathrm d I}{\mathrm d t}$$
\item CR$$ E = \frac{Q}{c} + R\frac{\mathrm d Q}{\mathrm d t}$$
\end{enumerate}


\begin{example}
An energy source with $100 V$ is connected in a series with a $10\Omega$ resistor and an inductor of $2H$. If the switch is closed at $t=0$, what is the current $I(t)$?
\end{example}

We note that substituting the values into the equation, we get $$100=10I+2\frac{\mathrm d I}{\mathrm d t}$$
Thus, by determining the integrating factor, we get

\begin{align*}
I' +5I &= 50\\
\left(Ie^{5t}\right)' &= \int 50e^{5t}\\
Ie^{5t} &= \int 50e^{5t}\\
&= 10e^{5t} + C\\
I &= 10+Ce^{-5t}
\end{align*}

Solving this, we find that $I(t) = 10 + Ce^{-5t}$. We note that at $t=0$, we have $I(0) = 0$. Thus, we find that $C=-10$, so the equation becomes 

$$I(t) = 10\left(1-e^{-5t}\right)$$

\section{September 30, 2016}
\subsection{Electric Circuits Cont'd}




\begin{example}
An energy source with $E=200e^{-50t}$ is connected in a series with a $2\Omega$ resistor and a $0.001 farad$ capacitor. If the initial charge on the capacitor is $0$, find the charge and the current at time $t>0$ and the maximal charge. 
\end{example}

We know that $E=E_R + EC$. However, we know that $E_R=2I$, and $E_C = \frac{Q}{0.01}$. Thus


\begin{align*}
E &= 2I + \frac{Q}{0.01}\\
&= 2Q' + 100Q\\
200e^{-50t} &= 2Q'+100Q\\
100e^{-50t} &= Q' + 500Q\\
Q'e^{50t} + 50Qe^{50t} &= 100e^{-50t}e^{50t}\\
Qe^{50t} &= \int 100\\
&= 100t+C\\
Q(t) &= (100t+C)e^{-50t}\\
\end{align*}

We note that at $t=0$, we have $C = 0$. Therefore, $Q(t) = 100te^{-50t}$. 

$I(t) = Q'(t)$, so it is equal to $100e^{-50t}-50*100te^{-50t}$, or alternatively $(1-50t)100e^{-50t}$.  $I(t)=0$ for $50t-1$, so $t=\frac{1}{50}$. We note that substituting this into $Q$, we get

\begin{align*}
Q\left(\frac{1}{50}\right)&= 100*\frac{1}{50}e^{-50*\frac{1}{50}}\\
&= 2e^{-1}\\
&= \frac{2}{e}\\
\end{align*}

Therefore, the charge is around $0.73576$ coulombs. 

\subsection{Second Order Linear Ordinary Differential Equations}
$$\frac{\mathrm d ^2y}{\mathrm d t^2} = q$$
$$F = -kx = m\frac{\mathrm d ^2x}{\mathrm d t^2}$$

$$y''(t) + p(t)y'(t) + q(t) y  =0$$ is the general form of a second order linear homogeneous ordinary differential equation. 

\begin{remark}
If $y(t)$ is a solution of the above equation, then $cy$ is also a solution of the above equation. If $y_1$ and $y_2$ are solutions, then by solving a system of equations, we can note that $y_1+y_2$ is also a solution. 
\end{remark}

Can we write the general solution as $y = C_1y_1 + C_2y_2$?

\begin{definition}
The functions $y_1$ and $y_2$ are \textbf{linearly independent} on interval $I$ if $a_1y_1(t) + a_2y_2(t) = 0$ for any $t \in I$ only for $a_1=a_2=0$
\end{definition}

\begin{example}
Are the following functions linearly dependent?

$$y_1(t) =e^t, y_2(t) = e^{2t}, t\in (-\infty, \infty)$$
$$y_1(t) = \ln(t), y_2(t) = \ln\left(t^5\right), t \in (0, \infty)$$
\end{example}

For the first equation, we note that we find $$ae^t + be^{2t} = 0$$. Substituting for $t=0$ and $t=1$, we find that $a=-b$. Thus, it is 

For the second equation, $y_2=5\ln(t) = 5y_1$. Thus, $5y_1-y_2=0$ for ant $t>0$.




\section{October 3, 2016}
\begin{definition} Functions $f_1, f_2, ..., f_n$ are \textbf{linearly independent} if $$a_1f_1(x) + a_2f_2(x) + ... + a_nf_n(x) = 0$$ for any $x \in I$ only for $a_1 = 0$, $a_2 = 0$, ... $a_n = 0$. The functions are \textbf{linearly dependent} if there are $a_1, a_2, ..., a_n$ where $|a_1| + |a_2| + ... + |a_n| > 0$ such that the above relation is equal to zero for any $x \in I$. 
\end{definition}




\begin{example}
Are the following equations linearly independent?

$$e^x, e^{2x}$$ 
$$\ln(x), \ln\left(x^5\right)$$
\end{example}

We note that the first set of equations are linearly independent on $\mathbb R$. $\ln(x)$ and $\ln\left(x^5\right)$ are linearly dependent on $(0,\infty)$ since $5\ln(x) +(-1)\ln\left(x^5\right) = 0.$


If $x_1(t)$ and $x_2(t)$ are solutions to the equation
$$y'' + p(t) y' + q(t)y = 0$$
then $c_1x_1 + c_2x_2$ is also a solution.  

\begin{definition}
If $x_1, x_2$ are linearly independent solutions of $y'' + p(t) y' + q(t) y = 0$, they form a \textbf{fundamental set of solutions}.
\end{definition}

Given that $$y = c_1x_z + c_2x_2$$ is a solution, then $$y' = c_1x_1' + c_2x_2'$$ Can we satisfy any initial set of conditions $y(0) = A, y'(0) = B$?

That is, the following set of conditions hold

$$y(0) = c_1x_1(0) + c_2x_2(0) = A$$
$$y'(0) = c_1x_1'(0) + c_2x_2'(0) = B$$

We can find a unique $\{c_1,c_2\}$ if the determinant is not equal to zero

$$\begin{vmatrix}
X_1(0) & X_2(0) \\ 
X_1'(0) & X_2'(0) \\ 
\end{vmatrix} \neq 0$$

\begin{definition}
Let $X_1, X_2$ be functions. Then the above determinant, denoted as $W[X_1, X_2](t)$, is call the \textbf{Wronskian}.
\end{definition}

If $y_1$ and $y_2$ are linearly dependent, then for any $t \in I$, we have the following three conditions

$$a_1y_1(t) + a_2y_2(t) = 0$$
$$a_1y_1'(t) + a_2y_2'(t) = 0$$
$$W[y_1, y_2](t) = 0$$

\begin{example}
Determine whether the following functions linearly dependent.

$$e^{2t}, e^t, t\in I$$
$$\cos(t) , \sin(t), t\in I$$


\end{example}

We evaluate the Wronskian of each pair of functions to find that 

\begin{align*}
\begin{vmatrix}
e^{2t} & e^t \\ 
2e^{2t} & e^t
\end{vmatrix} &= e^{2t}*e^t - e^t*2e^{2t}\\
&= e^{3t} - 2e^{3t} \\
&= -e^{3t} \\
\end{align*}
This is not equal to $0$ for any $t$, so it is linearly independent. For the second set of functions, we note that 

\begin{align*}
\begin{vmatrix}
\cos(t) & \sin(t) \\
-\sin(t) & \cos(t)
\end{vmatrix} &= \cos^2(t) + \sin^2(t) \\
&= 1
\end{align*}

This is also not equal to $0$, so it is also linearly independent. 

If we have an initial point$t$, we usually choose fundamental solutions satisfying
 $$X_1(t_0) = 1$$
$$ X_1'(t_0) = 0$$ 
$$ X_2(t_0) = 0$$
$$X_2'(t_0) = 1$$

\begin{theorem}[Existence and Uniqueness]
If $p$ and $q$ are continuous on an open interval $i$, $t_0 \in I$, then for any initial conditions
$$y(t_0) = A$$
$$y'(t_0) = B$$
, the initial value problem, concerning $y'' + p(t) y' + q(t)y = 0$, $y(t_0) = A$ and 
$y'(t_0) = B$, have a unique solution.
\end{theorem}

\begin{theorem}[Superposition Principle]
If $y_1$ and $y_2$ are solutions of $y'' + p(t) y' + q(t)y = 0$, then for any $C_1, C_2 \in \mathbb R$
$$y=C_1y_1 + C_2y_2$$
is a solution of $y'' + p(t) y' + q(t)y = 0$.
\end{theorem}

\begin{theorem}
If the Wronskian of $y_1$ and $y_2$ is nonzero at some point $t \in I$, then $y_1$ and $y_2$ are linearly independent on $I$. 
\end{theorem}

\begin{theorem}[Abel's Theorem]
$$W[y_1, y_2](t) = W(t_0)e^{-\int_{t_0}^tp(s) \mathrm d s}$$
\end{theorem}

\begin{proof}
We know that $$W(t) = y_1y_2' - y_1' y_2$$ and thus 
\begin{align*}
	W'(t) &= y_1'y_2' + y_1y_2''-y_1''y_2-y_1'y_2' \\
	&= y_1y_2'' - y_1''y_2
\end{align*}

We now compute 
\begin{align*}
	W'(t) + pW(t) &= y_1y_2'' -y_1''y_2+p\left(y_1y_2' - y_1'y_2\right)\\
	&= y_1y_2'' - y_1''y_2 + py_1y_2' - py_1'y_2 \\
	&= y_1\left(y)2''+py_2'\right) - y_2\left(y_1''+py_1'\right)\\
	&= y_1(-qy_2)-y_2(-qy_1)\\
	&= -qy_1y_2+qy_1y_2\\
	&= 0
\end{align*}

This means that both 
$$y_1'' + py_1' + qy_1 = 0$$
$$y_2'' + py_2' + qy_2 = 0$$

Thus we have
\begin{align*}
W(t)' + pW(t) &= 0\\
\frac{\mathrm d W}{\mathrm d t} &= -pW\\
\int \frac{\mathrm d W}{\mathrm W}  &= \int -p(t) \mathrm d t\\
\ln(W) &= C_1-\int p(t)\mathrm d t\\
|W| &= e^{C_1}e^{-\int p(t) \mathrm d t}\\
W(t) &= Ce^{-\int p(t) \mathrm d t} \\
\end{align*}

This if $W(t_0)$ is known, we have $$W[y_1, y_2](t) = W(t_0)e^{-\int_{t_0}^tp(s) \mathrm d s}$$
\end{proof}

\subsection{October 5, 2016}
\subsection{Second Order Linear Ordinary Differential Equations Cont'd}

\begin{remark}
$W[y_1, y_2](t)$ is either nonzero for any $t$, or $W(t) = 0$. 
\end{remark}

\begin{example}
Find the Wronskian of the Bessel equation given that $y_1(1) = 3, y_1'(1)=1, y_2(1)=2$, and $y_2'(1) = 2$.

$$x^2y'' + xy' + \left(x^2 -\alpha^2\right)y = 0$$
\end{example}

First, we note that $p(x) = x/x^2 = 1/x$ from the terms in front of the derivatives of $y$. We use Abel's theorem and find that 

\begin{align*}
W(1) &= \begin{vmatrix}
3 & 2 \\
1 & 2
\end{vmatrix}\\
&= 3*2-2*1\\
&= 4
\end{align*}

Now, we note that 

\begin{align*}
\int_1^x p(s) \mathrm d s &= \int_1^x \frac{1}{s}\mathrm d s\\
&= \ln(x) -\ln(1)\\
&= \ln(x)
\end{align*}

Thus, we have 

\begin{align*}
e^{-\ln(x)} &= \left(e^{\ln(x)}\right)^{-1}\\
&= x^{-1}\\
&= \frac{1}{x}
\end{align*}

Thus

\begin{align*}
W[y_1, y_2](x) &= W(1)*\frac{1}{x}\\
&= \frac{4}{x}
\end{align*}

Consider also a non-homogeneous equation 

$$y'' + p(t)y' + q(t)y = r(t)$$

\begin{theorem}
If $y_{part}$ is a particular solution of the above equation, then the general solution of the above equation is the general solution of $y'' + p(t)y' + q(t)y = 0$ + $y_{part}$. If $\{y_1, y_2\}$ is a fundamental set of solutions of $y'' + p(t)y' + q(t)y = 0$, then the general solution of $y'' + p(t)y' + q(t)y = r(t)$ is

$$y = C_1y_1 + C_2y_2 + y_{part}$$
\end{theorem}

\subsection{Reduction of Order}
$$y'' = -g$$

If a second order equation includes $y'''$ and $y'$ but not $y$, the substitution of $y' = u$ brings it to a first order equation. If we know a solution of $y_1$ of $y'' + p(t)y' + q(t)y = 0$, we can look for the general solution in the form $$y(t) = y_1(t)z(t)$$
Substituting, we get an equation in $z'', z'$.

\begin{example}
Solve the following equation knowing that $y_1 = t^{-1}$ is a solution
 $$t^2y'' + 3ty' + y = 0$$
\end{example}

Employing the theorem, we note that $y=\frac{1}{t}z$. Thus, $y' = -\frac{1}{t^2}z+ \frac{1}{t}z'$ and $y'' = \frac{2}{t^3}z + \frac{1}{t}z''$. Now, we substitute these values into the equation to get

\begin{align*}
t^2y'' + 3ty' + y &= t^2\left(\frac{2}{t^3}z + \frac{1}{t}z''\right) + 3t\left(-\frac{1}{t^2}z+ \frac{1}{t}z'\right) + \left(\frac{1}{t}z\right)\\
&= tz'' + z'\\
0 &= tz'' + z
\end{align*}

We now make the substituting of $u =z'$ and $u' = z''$. Thus, we get

\begin{align*}
t\frac{\mathrm d u}{\mathrm d t} &= -u\\
\int\frac{\mathrm d u}{u} &= -\int\frac{\mathrm d t}{t}\\
\ln(u) &= C_1-\ln(t)\\
u &= \pm e^{C_1}\frac{1}{t}\\
&= \frac{C}{t}
\end{align*}

Now, we substitute back to find $z'$ and $z''$

\begin{align*}
z(t) &= \int u(t) \mathrm d t\\
&= \int \frac{C}{t} \mathrm d t\\
&= C_2\ln(t) + C_1
\end{align*}

We recall that $y (t) = \frac{1}{t}z$. Therefore

$$y(t) = t^{-1}\left(C_1 + C_2\ln(t)\right)$$

\subsection{Equations with Constant Coefficients}

These equations are of the form $$ay'' + by' + cy = 0$$
To solve equations of this form we look for a solution of the form $y = e^{rt}$, $y' = re^{rt}$ and $y'' = r^2e^{rt}$. The equation becomes

$$ar^2e^{rt} + bre^{rt} + ce^{rt} = 0$$
However, we note that $e^{rt} \neq 0$, so we solve the quadratic equation of

$$ar^2+br+c=0$$
This is known as the \textbf{characteristic equation} of $ay'' + by' + cy = 0$.

We note that if $b^2-4ac > 0$, and there are two distinct real roots $r_1 \neq r_2$, then the general solution of $ay'' + by' + cy = 0$ is 

$$y = C_1e^{r_1t} + C_2e^{r_2t}$$

\begin{example}
Given $y(0) = 7$ and $y'(0)=-8$, solve the initial value problem for the following equation
$$y'' -2y'-8y=0$$
\end{example} We immediately write the characteristic equation to get $$r^2-24-8=0$$
Solving this equation, we get 
\begin{align*}
r_{1,2} &= \frac{2 \pm \sqrt{2^2 + 32}}{2}\\
&= 1 \pm 3
\end{align*}
Therefore, we denote $r_1=-2$ and $r_2 = 4$. The general solution is therefore
$$y=C_1e^{-2t}+C_2e^{4t}$$
Thus, $y' = -2C_1e^{-2t} + 4C_2e^{4t}$. We now substitute the initial values for $y(0)$ and $y'(0)$ to get that $C_1 = 6$ and $C_2 = 1$. Therefore
$$y = 6e^{-2t}+e^{4t}$$

\section{October 7, 2016}
\subsection{Constant Coefficients}

Recall the general formula for solving second order ordinary differential equations with constant coefficients
$$ay'' + by' + cy=0$$
If we let $y = e^{rt}$, then this becomes
$$ar^2+br+c=0$$
which is the characteristic equation. We note that there are three cases

\begin{enumerate}
	\item Case 1, $b^2-4ac > 0$:
	The equation has two real roots where $r_1 \neq r_2$. The general solution is $$y = C_1e^{r_1t} + C_2e^{r_2t}$$
	\item Case 2, $b^2-4ac =0$:
	The equation has two equal roots where $r=-\frac{b}{2a}$. A solution is $$y_1 = e^{rt}$$ Let us use the ``reduction of order" method to get
	$$y = e^{rt}z(t)$$
	$$y' = re^{rt}z + e^{rt}z'$$
	$$y'' = r^2e^{rt}z + ere^{rt}z' + e^{rt}z''$$
	Substituting into $ay'' + by' + cy=0$, we get
	\begin{align*}
		ay'' + by' + cy=0 &=a\left(r^2e^{rt}z + 2re^{rt}z' + e^{rt}z''\right) + b\left(re^{rt}z + e^{rt}z'\right) + ce^{rt}z \\
		&= e^{rt}\left(ar^2z+2arz'+az'' + brz"bz'+cz\right)\\
		&= e^{rt}\left(\left(ar^2+br+c\right)z + (2ar+b)z' + az''\right)\\
		&= e^{rt}\left((0)z + (0)z' + az''\right)
	\end{align*}
	
	Thus, since $az''=0$, we known that $z'' = 0$. Therefore, $z=C_1+C_2t$. The general equation  is $$y=e^{rt}\left(C_1+C_2t\right)$$
	\item Case 3, $b^2-4ac < 0$: The equation has two complex roots of the form $r_{1,2} = \alpha + \beta i$. Specifically, they are
	$$r = -\frac{b}{2a} \pm \frac{\sqrt{4ac-b^2}}{2a}i$$
	We employ Euler's formula to get 
	$$y_1  =e^{\alpha t + \beta i t} = e^{\alpha t}\left(\cos(\beta t) + i\sin(\beta t)\right)$$
	$$y_2  =e^{\alpha t - \beta i t} = e^{\alpha t}\left(\cos(\beta t) - i\sin(\beta t)\right)$$
	We now consider $X_1 = \frac{y_1+y_2}{2} = e^{\alpha t}\cos(\beta t)$, and $X_2 = \frac{y_1-y_2}{2i} = e^{\alpha t}\sin(\beta t)$, which form a fundamental set of solutions. The general solution is therefore $$y=e^{\alpha t}\left(C_1\cos(\beta t) + C_2\sin(\beta t)\right)$$
\end{enumerate}

\begin{example}
Find the general solution to $$y'' +-6y'+9y=0$$
\end{example}

We use the characteristic equation and solve $r^2-6r+9=0$ to get
$$r = \frac{6 \pm \sqrt{6^2-9*4}}{2} = \frac{6 \pm 0}{2} = 3$$
The general solution is therefore $$y=\left(C_1 + C_2t\right)e^{3t}$$

\begin{example}
Solve $$y'' -6y'+10y=0$$
\end{example}

We note that the characteristic equation is $$r^2-6r+10=0$$ Solving for $r$, we get $r = 3\pm i$. Therefore, the general solution is $$y = e^{3t} \left(C_1\cos(t) + C_2\sin(t)\right)$$

\begin{remark}
We may have an initial value problem with $y(t_0) = A, y'(t_0) = B$. A unique solution exists. 
\end{remark}

For equations of the form $ay'' + by' + cy =0$, we may have a bounded value problem with $y(t_0) = A, y(t_1) = B$. 

\begin{example}
	Solve the boundary value problem for the following equations and boundaries
	$$y'' + 4y =0, y(0) = 0, y\left(\frac{\pi}{2}\right) = 1$$
	$$y'' + 4y = 0, y(0) = 0, y\left(\frac{\pi}{2}\right) =0$$
\end{example}

For the first equation and boundaries, we find that the characteristic equation is $r^2+4=0$. We note that solving for $r$, we get $r=\pm 2i$. Since there is no real part, the general solution is therefore 
$$y=C_1\cos(2t) + C_2\sin(2t)$$
We now note that substituting the initial values, we obtain 
$$y(0) = C_1\cos(0) + C_2*0 = C_1 = 0$$
$$y\left(\frac{\pi}{2}\right)  = C_1\cos(\pi) + C_2 \sin(\pi) =- C_1 = 1$$
Thus, there are no solutions.

For the second equation and boundaries, we substitute the initial values to find that
$$y(0) = C_1 = 0$$
$$y\left(\frac{\pi}{2}\right) = -C_1 = 0$$
We note in this case that $C_2$ can be any value, so the equation becomes $$y = C\sin(2 t)$$
Thus, there are infinitely many solutions.

\section{October 12, 2016}
\subsection{Higher Order Linear Ordinary Differential Equations}

\begin{example}
Solve the second order boundary value problem $$y'' -4y = 0, y(0) = 0, y'\left(\frac{\pi}{2}\right) = 2$$
\end{example}
We note that the characteristic equation becomes $$r^2-4=0$$ We factor this to find that $r = \pm 2$. The general solution is therefore 
$$C_1e^{2t} +C_2e^{-2t}$$Thus, substituting the initial values we have $y(0) = C_1 + C_2 = 0$ and $y'\left(\frac{\pi}{2}\right) = C_1e^{\pi} + C_2e^{-\pi} = 1$. We make use of Cramer's rule to find

$$C_1 = \frac{\begin{vmatrix}0 & 1\\ 1 & e^{-\pi}\end{vmatrix}}{\begin{vmatrix} e^0 & e^0 \\ e^{\pi} & e^{-\pi}\end{vmatrix}}$$

$$C_2 = \frac{\begin{vmatrix}1 & 0\\  e^{\pi} & 1\end{vmatrix}}{\begin{vmatrix} 1 & 1 \\ e^{\pi} & e^{-\pi}\end{vmatrix}}$$
It is convenient to use 
$$\cosh (kt) = \frac{e^{kt} + e^{-kt}}{2}$$
$$\sinh(kt) =  \frac{e^{kt} -e^{-kt}}{2}$$
We can then re-write the general solution as 
$$y(t) = C_1\cosh(2t) + C_2\sinh(2t)$$
Thus, $y(0) = C_1*1 + C_2*0 = C_1 = 0$ and $y\left(\frac{\pi}{2}\right) = C_2\sinh\left(2*\frac{pi}{2}\right) = 1$, so $C_2 = \frac{1}{\sinh(\pi)}$. 


\subsection{Higher Order Differential Equations}
We now proceed to consider higher order linear equations. Consider equations of the form 
$$y^{(n)}  + a_{n-1}(t)y^{(n-1)} + ... + a_1(t)y' + a_0(t)y = 0$$
where $$y(0) = y_0, ..., y^{(n-1)}(0) = y_0^{(n-1)}$$ We require the following:

\begin{enumerate}
	\item If $a_{n-1}, ..., a_1, a_0$ continuous on an interval $I$, including the initial point $t_0 = 0$, then both of the above equations has a unique solution on $I$.
	\item If $y_1, y_2, ..., y_k$ are solutions of the first equation above, then $y = C_1y_1 + ... + C_k y_k$ is a solution of the first equation above for any choice of $C_1, ..., C_k$ (superposition principle). 
	\item We define the Wronskian of $y_1, ..., y_n$ as 
	
	$$W[y_1, ..., y_n](t) = \begin{vmatrix}y_1 & y_2 & ... & y_n\\
	y_1' & y_2' & ... & y_n' \\
	... & ... & ... & ... \\
	y_1^{(n-1)} & y_2^{(n-1)} & ... & y_n^{(n-1)}
	\end{vmatrix}$$
\end{enumerate}
Thus $$y(t) = C_1y_1 + C_2y_2 + ... +C_ny_n$$ is the general solution of the first of the above equations if $y_1, y_2, ..., y_n$ are linearly independent, of $W(t) \neq 0$. In fact, for solutions of the first equation above, either $W(t) = 0$ or $W(t) \neq 0$ at any $t$. 

\begin{example}
Find the linearly independent solutions of $$y''' + y' = 0$$ and show that they are independent.
\end{example}

We use the ``reduction of order" and denote $y' = z$. Thus, $y''' = z''$. The equation now becomes $$z'' + z = 0$$ The characteristic equation becomes $r^2+1=0$, so $r = \pm i$. Thus we have $z_1(t) = \cos(t)$, $z_2(t) = \sin(t)$ and $z = C_1\cos(t) + C_2\sin(t)$. Thus
$$y(t) = \int z\mathrm d t = C_1\sin(t) -C_2\cos(t) + C_3$$
We therefore have $1, \sin(t)$, and $\cos(t)$ as our three solutions, which we denote $y_1, y_2$, and $y_3$ respectively. To show that they are independent, we note that 

\begin{align*}
	W[y_1, y_2, y_3](t) &= \begin{vmatrix}1 & \sin(t) & \cos(t) \\ 0 & \cos(t) & -\sin(t) \\ 0 & -\sin(t) & -\cos(t)\end{vmatrix}\\
	&= \begin{vmatrix}\cos(t) &-\sin(t) \\ -\sin(t) & -\cos(t) \end{vmatrix}\\
	&= -\cos^2(t) -\sin^2(t) \\
	&= -1 \neq 0
\end{align*} 
Thus, $y_1, y_2$, and $y_3$ are linearly independent. 

We note that to solve the equation
$$y^{(n)}  + a_{n-1}y^{(n-1)} + ... + a_1y' + a_0y = 0$$
we write out and solve the characteristic equation 
$$r^n + a_{n-1}r^{n-1} + ... + a_1r + a_0 = 0$$

\begin{example}
Solve $$y''' -27y = 0$$
\end{example}

We note that the characteristic is $r^3 -27 = 0$. We note one solution is $r_1 = 3$. Factoring $(r-3)$ out of $r^3-27$, we get $r^2+3r+9$. We now solve this using the quadratic equation to determine the complex roots. The quadratic equation becomes $$\frac{3 \pm \sqrt{3^2 -4*9}}{2}$$
Solving this, we get $r_{2,3} = -\frac{3}{2} \pm \frac{3\sqrt{3}}{2}i$. Thus, the general solution is $$y = C_1e^{3t} +e^{-\frac{3}{2}t}\left(C_2\cos\left(\frac{3\sqrt{3}}{2}t\right) + C_3\sin\left(\frac{3\sqrt{3}}{2}t\right)\right)$$



\section{October 14, 2016}
\subsection{Higher Order Linear Differential Equations with Constant Coefficients}

We recall that for equations of the form 
$$y^{(n)}  + a_{n-1}y^{(n-1)} + ... + a_1y' + a_0y = 0$$
we may write out the characteristic equation 
$$r^n + a_{n-1}r^{n-1} + ... + a_1r + a_0 = 0$$
Note that for the characteristic equation, we can have $r_1$ real, $r_1 = r_2 = ... = r_k$ real, $r_1 = \alpha \pm \beta i$ and $r_1 = r_2 = \alpha \pm \beta i$. Their respective solutions are therefore $$e^{r_1t}$$ $$e^{r_1t}, te^{r_1t}, ..., t^{k-1}e^{r_1t}$$ $$e^{\alpha t}\cos(\beta t), e^{\alpha t}\sin(\beta t)$$ $$e^{\alpha t}\cos(\beta t), e^{\alpha t}\sin(\beta t), te^{\alpha t}\cos(\beta t), te^{\alpha t}\sin(\beta t)$$

\begin{example}
Does there exist a linear homogeneous ordinary differential equation with constant coefficients which has a solution of $$y = t^2e^{3t} + te^{-t}$$ of the fourth order. Of the fifth order?
\end{example}

We note that there are linearly independent solutions of $e^{3t}, te^{3t}, t^2e^{3t}$ and $e^{-t}, te^{-t}$. Thus, there is none for the fourth order, but there is for the fifth order. Namely, the equation $$(r+1)^2(r-3)^3=0$$

\begin{example}
Solve $$y^{(4)}-3y''-4y = 0$$
\end{example}
The characteristic equation is $r^4-3r^2-4=0$. We make the substitution of $t=r^2$. Solving the resulting equation, we find that $$t = \frac{3 \pm \sqrt{9 + 16}}{2}$$ Thus, we have $t_1 = 4, t_2 = -1$. Thus, we have $r = \pm 2, \pm i$. The general solution is $y = y(x)$, so $$y = C_1e^{2x} + C_2e^{-2x} + C_3\cos(x) + C_4\sin(x)$$

We now consider equations of the form $$y^{(n)}  + a_{n-1}y^{(n-1)} + ... + a_1y' + a_0y = f(t)$$
\begin{theorem}
A general solution of the above equation is a general solution of $y^{(n)}  + a_{n-1}y^{(n-1)} + ... + a_1y' + a_0y = 0$ and a particular solution of $y^{(n)}  + a_{n-1}y^{(n-1)} + ... + a_1y' + a_0y = f(t)$.
\end{theorem}

\subsection{The Method of Undetermined Coefficients}

\begin{example}
Find the general solution of $$y'' + y' + 2y = 2x^2$$
\end{example}

We first solve the characteristic equation of $y'' + y' + 2y = 0$. Thus, we have $r^2+r+2=0$. Solving this, we find that $$r = \frac{-1 \pm \sqrt{1-8}}{2} = -\frac{1}{2} \pm \frac{\sqrt{7}}{2}i$$ Thus we have $$y_{homogeneous} = e^{-\frac{x}{2}}\left(C_1\cos\left(\frac{\sqrt{7}x}{2}\right) + C_2\sin\left(\frac{\sqrt{7}x}{2}\right) \right)$$


Now, note that $y_{particular} = Ax^2 + Bx + C$. Thus, $y' = 2Ax + B$ and $y'' = 2A$. That is, we can equate the following in terms of $y'' + y' + 2y$ to obtain
$$2A + 2Ax + B + 2\left(Ax^2+Bx + C \right)= 2x^2$$ We find that $A = 1, B = -1$, and $C  = -\frac{1}{2}$. Thus we have $$y_{particular} = x^2-x-\frac{1}{2}$$ The general solution is therefore 

\begin{align*}
y &= y_{homogeneous} + y_{particular} \\
&=  e^{-\frac{x}{2}}\left(C_1\cos\left(\frac{\sqrt{7}x}{2}\right) + C_2\sin\left(\frac{\sqrt{7}x}{2}\right) \right) + x^2-x-\frac{1}{2}
\end{align*}


\subsection{Summary of Solutions}

We note that if the right hand side is the following, then the solutions are of the form:

\begin{itemize}
	\item $$P_n(x) \rightarrow Q_n(x)$$
	\item $$\sin(\beta x) \text{ or } \cos(\beta x) \rightarrow A\sin(\beta x) + B\cos(\beta x)$$
	\item $$P_n(x)e^{kx} \rightarrow Q_n(x)e^{kx}$$
	\item $$P_n(x)e^{kx}\cos(\beta x) \text{ or } P_n(x)e^{kx}\sin(\beta x)\rightarrow e^{kx}\left(Q_n(x)\cos(\beta x) + R_n(x)\sin(\beta x)\right)$$
\end{itemize}

\section{October 17, 2016}
\subsection{Undetermined Coefficients Method Cont'd}

\begin{example}
Solve $$y'' -4y'=3\cos(t)$$
\end{example}

We use the characteristic equation to get $r^2-4r=r(r-4) = 0$. That is, $r_{1,2} = 0, 4$. The homogeneous equation is thus given by $$y_{homogeneous} = C_1e^{0t} + C_2e^{4t}$$
Now, we have $y_{particular} = A\cos(t) + B\sin(t)$. Thus, $y' = -A\sin(t) + B\cos(t)$ and $y'' = -A\cos(t) -B\sin(t)$. The initial equation is then of the form $$y'' -4y'  = -A\cos(t)-B\sin(t) + 4A\sin(t) -4B\cos(t) = 3\cos(t)$$ Separating terms into $\cos$ and $\sin$ and solving for $A,B$, we find that $A = -\frac{3}{17}$ and $B = -\frac{12}{17}$. The particular solution is therefore $$y_{particular} = -\frac{3}{17}\cos(t) -\frac{12}{17}\sin(t)$$ The general solution is $y = y_{homogeneous} + y_{particular}$. Therefore, the general solution is $$y = C_1 + C_2e^{4t} -\frac{3}{17}\cos(t) - \frac{12}{17}\sin(t)$$

\begin{example}
Solve $$y'' -4y' = t+2$$
\end{example}

We note that the characteristic equation is the same as the previous example. Thus, we have $$y_{homogeneous} = C_1 + C_2e^{4t}$$ For the particular solution, if the right hand side is a polynomial $P_n(t)$ and $r = 0$ is a root of the characteristic equation ($s$ times), we multiply the proposed solution by $t^s$. In this case, $0$ is a root one time, so $y_{particular} = t(At+B) = At^2+Bt$. Thus, $y' = 2At + B$ and $y'' = 2A$. Now, substituting the original equation, we have $$y'' -4y' = 2A-8At -4B = t+2$$ We now compare the values to solve for $A$ and $B$. We find that $A  = -\frac{1}{8}$ and $B = -\frac{9}{16}$. Therefore $$y = C_1 + C_2e^{4t} +t\left(-\frac{1}{8}t - \frac{9}{16}\right)$$

\subsection{Summary of Solutions}

We note that by applying the undetermined coefficients method, we have the following rules. 
\begin{enumerate}
	\item When the right hand side is of the form $$P_n(t)e^{kt}$$ where $r=k$ is not a solution of the characteristic equation, the proposed solution is $$Q_n(t)e^{kt}$$
	\item When the right hand side is of the form $$P_n(t)e^{kt}$$ where $r=k$ is a root $s$ times, the proposed solution is $$t^sQ_n(t)e^{kt}$$
\end{enumerate}

\begin{example}
Solve $$y'' + 9y = -18\cos(3t)$$
\end{example}

We first determine that the characteristic equation is $r^2+9=0$. Solving this, we find that $r = \pm 3i$. Therefore the homogeneous solution is $$y_{homogeneous} = C_1\cos(3t) +C_2\sin(3t)$$ We now want to find the particular solution. We now multiply the general form of the particular solution by $t$ once since each $r$ is a solution only once. That is, we have $y_{particular} = At\cos(3t) + Bt\sin(3t)$. Thus, $y' = A\cos(3t) -3At\sin(3t) + B\sin(3t) + 2Bt\cos(3t)$ and $y'' = -3A\sin(3t) -3A\sin(3t) -9At\cos(3t) + 3B\cos(3t) + 3B\cos(3t) -9Bt\sin(3t)$. We now substitute after collecting like terms to find that 
\begin{align*}
	y'' +9y &= -6A\sin(3t) -9At\cos(3t) +6B\cos(3t) - 9Bt\sin(3t) + 9At\cos(3t) + 9Bt\sin(3t)\\
	&= -6A\sin(3t) + 6B\cos(3t) \\
	&= -18\cos(3t) 
\end{align*}
We now equate the last two expression and find that $A = 0$ and $B = -3$. Therefore, we find that $$y_{particular} = -3t\sin(3t)$$ We now determine the general solution by adding the homogeneous and particular solutions to get $$y = C_1\cos(3t) + C_2\sin(3t) -3t\sin(3t$$

\begin{example}
Determine which form $y_{particular}$ should be in for the following three equations using the undetermined coefficients method.
$$y^{(5)} -y' = t\cos(t)$$
$$y^{(5)} -y' = e^{2x} + e^{-x} + xe^x$$
$$y^{(5)} -y' = x^3e^{5x}$$ 
\end{example}

We determine the characteristic equation to be $r^5-r=r\left(r^4-1\right) = r\left(r^2-1\right)\left(r^2+1\right)$. The roots are therefore, $r = 0, \pm 1, \pm i$. By apply this, we note that for the first equation, we have $$y_{particular} = t(A+Bt)\cos(t) + t(C + Dt) \sin(t)$$ For the second equation, we note that $r_{2,3} = 1, -1$ so we multiply the third and second term respectively by $x$ (since $1, -1$ match the exponents above $e$) to get $$y_{particular} = Ae^{2x} + Bxe^{-x} + x(Cx + D)e^x$$ For the third equation, we have $$y_{particular} = \left(Ax^3 + Bx^2 + Cx + D\right)e^{5x}$$

\section{October 19, 2016}
\subsection{Variation of Parameters}

We solve the homogeneous equation 
$$y'' + p(t)y'  + q(t)y = 0$$ to find the solution below, where $y_1, y_2$ are functions $$y = C_1y_1 + C_2y_2$$ We therefore assume that equations of the form $$y'' + p(t)y'  + q(t)y = r(t)$$ can be solved to find the solution $$y = C_1(t)y_1(t) + C_2(t)y_2(t)$$ To find $C_1(t)$ and $C_2(t)$, we need to solve the system $$C_1'y_1 + C_2'y_2 = 0$$ $$C_1'y_1' + C_2'y_2' = r(t)$$ In this case, we have $$C_1(t) = -\int \frac{r(t)y_2(t)}{W(t)}\mathrm d t + \bar{C_1}$$ $$C_2(t) = \int \frac{r(t)y_1(t)}{W(t)}\mathrm d t + \bar{C_2}$$ Thus, where $y = C_1(t)y_1(t) + C_2(t)y_2(t)$ is the general solution, we have a particular solution of $$y_{particular} = -y_1\int \frac{ry_2}{W}\mathrm d t + y_2 \int \frac{ry_2}{W}\mathrm d t$$


\begin{example}
Solve $$y'' + y = \frac{1}{\sin(t)}$$
\end{example}

The characteristic equation is $\lambda^2 + 1 = 0$. Solving this, we get $\lambda = \pm i$. Thus, we have $y_1 = \cos(t)$ and $y_2 = \sin(t)$. Similarly, finding the derivatives we have $y_1' = -\sin(t)$ and $y_2' = \cos(t)$. We know that our general solution is of the form $$y = C_1(t)\cos(t) + C_2(t)\sin(t)$$ We now solve the system of equations $$C_1'y_1 + C_2'y_2 = 0$$ $$C_1'y_1' + C_2'y_2' = r(t)$$ Doing this, we find that $C_2(t) = \ln(\sin(t)) + \bar{C_2}$ and $C_1(t) = -t + \bar{C_1}$. Thus, we have $$y = \bar{C_1}\cos(t) + \bar{C_2}\sin(t) -t\cos(t) + \sin(t)\ln(\sin(t))$$ after substituting $C_1(t)$ and $C_2(t)$. 

\begin{example}
$$y'' + 16y = e^{-t}$$ has a solution satisfying $\lim_{t \to \infty} y(t) = 0$. Find the solution, $y(0)$, and $y'(0)$. 
\end{example}

We note that the characteristic equation is $\lambda ^2 + 16 =0$. Thus, we have $\lambda = \pm 4i$. Therefore, we have $y_1 = \cos(4t)$ and $y_2 = \sin(4t)$. Our particular solution is of the form $y_p = Ae^{-t}$ and so $y' = -Ae^{-t}$ and $y'' = Ae^{-t}$. Now, we substitute this into the original equation of $y'' + 16y = e^{-t}$ to get 
$$Ae^{-t} + 16Ae^{-t} = e^{-t}$$ Solving this, we get $A = \frac{1}{17}$. Thus, our solution becomes $$y = C_1\cos(4t) + C_2\sin(4t) + \frac{1}{17}e^{-t}$$ We note that this tends towards 0 as $t \to \infty$, so $C_1, C_2 = 0$. Furthermore, we have $y(0) = \frac{1}{17}$ and $y'(0) = -\frac{1}{17}$. 

\begin{example}
$$y = 4te^{5t} + 7e^{3t}\cos(t)$$ is a solution of $$y^{(4)} + ay''' + by'' + cy' + dy = 0$$ Find $a, b, c, d$.
\end{example}

We note that the roots of the characteristic equation $$r^4 + ar^3 + br^2 + cr + d = 0$$ are $r_{1,2} = 5$, $r_{3,4} = 3 \pm i$. Multiplying this, we find that $\left(r^2-10r+25\right)\left(r^2-6r+10\right)$.

\section{October 21 , 2016}

\subsection{Introduction and Properties of Laplace Transform}

The \textbf{Laplace Transform} is defined for a piecewise continuous $f$ satisfying $|f(t)| \leq Me^{bt}$ where $M, b > 0$. That is, we have $$L[f](s) = F(s) = \int_{0}^{\infty} e^{-st}f(t) \mathrm d t $$


\begin{example}
Evaluate $$L[1] = \int_{-\infty}^{\infty}  e^{-st} \mathrm d t$$
\end{example}

\begin{align*}
	L[1] &= \lim_{A \to \infty} \int_0^Ae^{-st} \mathrm d t \\
	&= \lim_{A \to \infty} \left(-\frac{1}{s}e^{-st}|_{t=0}^{t=A}\right) \\
\end{align*}
For $s>0$, we have $L[1] = \frac{1}{s}$.

\begin{example}
Evaluate $$L[e^{at}]$$
\end{example}

\begin{align*}
L[e^{at}] &= \int_0^{\infty}e^{-st}e^{at} \mathrm d t\\
&= \int_0^{\infty} e^{-(s-a)t} \mathrm d t \\
&= -\frac{1}{s-a} \left(e^{-(s-a)t} -e^0 \right) \\
\end{align*}
For $s-a > 0$, we have $L[e^{at}] = \frac{1}{s-a}$


\subsection{Properties of Laplace Transform}

\begin{enumerate}
\item Linearity states that for constants $a,b$ and functions $f,g$ we have $$L[af + bg] = aL[f] + bL[g]$$
\item $$L[\sin(bt)] = \frac{b}{s^2+b^2}, s > 0$$
\item $$L[\cos(bt)] = \frac{s}{s^2 + b^2}, s > 0$$
\item If $f$ is piecewise continuous and $|f(t)| \leq Me^{bt}$, then $L[f](s)$ is defined for $s > b$. 
\item $$L[f'(t)] = L\left[\frac{\mathrm d }{\mathrm d t} f \right] = sL[f] -f(0)$$
\item The first differentiation formula is given as $$L\left[\frac{\mathrm d^n}{\mathrm d t^n}f\right] = s^nL[f] -s^{n-1}f(0) - ... -sf^{(n-2)}(0) - f^{(n-1)}(0) $$
\item We use the following to get $L[t^n]$, $$L[tf(t)] = -\frac{\mathrm d }{\mathrm d s} L[f](s)$$
\item The second differentiation formula is given as $$L[t^nf(t)] = (-1)^n \frac{\mathrm d^n}{\mathrm d s^n}L[f](s)$$
\end{enumerate}

\begin{example}
Compute the Laplace Transform of $$L[3 \cos(2t) + 5e^{-4t}]$$
\end{example}

\begin{align*}
	&= 3L[\cos(2t)] + 5L[e^{-4t}] \\
	&= \frac{3s}{s^2+b^2} + \frac{5}{s+4}
\end{align*}

\begin{example}
Find $$L[\sin^2(t)]$$
\end{example}

\begin{align*}
&= L \left[\frac{1-\cos(2t)}{2} \right] \\
&= \frac{1}{2}L[1] - \frac{1}{2} L[\cos(2t)] \\
&= \frac{1}{2s} - \frac{s}{2\left(s^2+4\right)}
\end{align*}






\section{October 24, 2016}
\subsection{The Laplace Transform Cont'd}

Given that $s > 0$, we recall the formula of the Laplace Transform $$L\left[f(t)\right](s) = \int_0^{\infty}f(t)e^{-st}\mathrm d t$$
The following properties hold 
\begin{enumerate}
	\item Linearity.
	\item $L[f](s)$ exists for $s > b$ if $|f(t)| \leq Me^{bt}$.
	\item $L\left[f'(t)\right](s) = sL[f] - f(0)$. (The First Differentiation Formula)
	\item $L[tf(t)](s) = -\frac{\mathrm d }{\mathrm d s} L[f](s)$.
	\item $L\left[e^{at}f(t)\right] = L[f](s-a)$ (The First Shift Formula)
	\item $L\left[\int_0^t f(w) \mathrm d w\right](s) = \frac{1}{s}L[f](s)$
	\item $L\left[U_a(t)g(t)\right] = e^{-as}L\left[g(t+a)\right]$
\end{enumerate}

\begin{proof}(Property 4)
Let us prove that the integral of both sides is equal. That is, we have 
\begin{align*}
\int_{s_0}^s -L[tf(t)](w) \mathrm d w &= L[f](s) -L[f](s_0) \\
&= -\int_{s_0}^s \mathrm d t \int_0^{\infty} tf(t)e^{-wt} \mathrm d w \\
&= \int_0^{\infty} \mathrm d t f(t) \int_{s_0}^s -te^{-wt}\mathrm d w \\
&= \int_0^{\infty}f(t) \mathrm d t \left[e^{-wt}\right]|_{w=s_0}^{w=s} \\
&= \int f(t) \mathrm d t \left(e^{-st} -e^{-s_0t}\right) \\
&= \int f(t)e^{-st} \mathrm d t - \int f(t) e^{-s_0t} \mathrm d t \\
&= L[f](s) -L[f](s_0)
\end{align*}
\end{proof}

\begin{example}
Solve $$L\left[te^t\right]$$
\end{example}
We note that this is 
\begin{align*}L[te^t] &= -\frac{\mathrm d }{\mathrm d s} L\left[e^t\right] \\
&= -\frac{\mathrm d }{\mathrm d s} \left(\frac{1}{s-1}\right) \\
&= \frac{1}{(s-1)^2}
\end{align*}

\begin{example}
Solve $$L\left[t^3\right]$$
\end{example}

\begin{align*}
	L[t] &= -\frac{\mathrm d }{\mathrm d s}L[1] = -\left(\frac{1}{s}\right)' = \frac{1}{s^2} \\
	L\left[t^2\right] &= -\frac{\mathrm d }{\mathrm d s} L[t] =  -\frac{\mathrm d }{\mathrm d s}\left(\frac{1}{s^2}\right) = \frac{2}{s^3} \\
	L\left[t^3\right] &=  -\frac{\mathrm d }{\mathrm d s} L\left[t^2\right] =  -\frac{\mathrm d }{\mathrm d s}\left(\frac{2}{s^3}\right) = \frac{6}{s^4}
\end{align*}

\begin{example}
Solve $$L\left[e^{at}\cos(bt)\right]$$
$$L\left[e^{at}\sin(bt)\right]$$
\end{example}

We use the first shift formula to find that the first evaluates to $$\frac{s-a}{(s-a)^2 + b^2}$$ and the second evaluates to $$\frac{b}{(s-a)^2+b^2}$$


\begin{example}
	Solve $$L\left[\int_0^t e^{8w}\cos(6w)\mathrm d w\right](s)$$
\end{example}

We apply the sixth property to find that 
\begin{align*}
	L\left[\int_0^t e^{8w}\cos(6w)\mathrm d w\right](s) &= \frac{1}{s}L\left[e^{8t}\cos(6t)\right] \\
	&= \frac{s-8}{s\left((s-8)^2+36\right)}
\end{align*}



\begin{example}
Solve $$L\left[e^{2t}\left(t^2-5t+6\right)\right]$$
\end{example}

We note that this is a shift of $2$ for $L\left[t^2-5t+6\right]$. Thus, we first solve this and then apply the shift to get
\begin{align*}
	L\left[t^2-5t+6\right] &= \frac{2}{s^3} -\frac{5}{s^2} + \frac{6}{s} \\
	L\left[e^{2t}\left(t^2-5t+6\right)\right]  &=  \frac{2}{(s-2)^3} -\frac{5}{(s-2)^2} + \frac{6}{s-2}
\end{align*}

The Laplace Transform can deal with discontinuous functions. Consider the step function $U_a$ which is defined as $$U_a(t) = \begin{cases} 0, 0 \leq t < a \\ 1, t \geq a \end{cases}$$

\begin{example}
	Solve $$L[U_a(t)](s) $$
\end{example}

We note that for $s>0$, we have 
\begin{align*}
	L[U_a(t)](s)  &= \int_0^{\infty}U_a(t)e^{-st} \mathrm d t \\
	&= \int_0^a0 \mathrm d t + \int_a^{\infty}1*e^{-st} \mathrm d t \\
	&= 0 -\frac{1}{s}e^{-st}|_{t=a}^{t=\infty} \\
	&= -\lim_{t \to \infty} \left(\frac{1}{s}e^{-st}\right) + \frac{1}{s}e^{-as} \\
	&= \frac{e^{-as}}{s}
\end{align*}

\section{October 26, 2016}
\subsection{Midterm Review}

\begin{example}
Find $y''$ given that $$y' + (2t+1)y = 2\cos(t)$$ and $y(0) = 2$.
\end{example}

We note that by rearranging the equation, we find that $y'(0) = 0$ when we substitute $y=2$ at $t =0$. We then differentiate the equation to get $$y'' + 2y + (2t+1)y' = -2\sin(t)$$ We can solve this to find that $y''(0) = -4$.

\begin{example}
Let $ay'' + by' + cy = 0$ and $y_{particular} = \cos(2x)$. What is the general solution?
\end{example}

From the particular solution, we note that $r = \pm 2i$. Thus, $$y_{general} = C_1\cos(2x) + C_2\sin(2x)$$

\begin{example}
	Solve $$y'' + e^ty' + y = t^3$$
\end{example}


\section{October 31, 2016}
\subsection{The Laplace Transform Cont'd}

\begin{example}
Compute $$\int_0^{\infty}e^{-3t}\cos(5t)\mathrm d t$$
\end{example}

Traditionally, we would integrate by parts and then compute the improper integral as a limit. However, we note that we can now solve this more simply using Laplace Transform. That is, we have 

\begin{align*}
L[f] &= L[f](s) \\
&= \int_0^{\infty}f(t)e^{-st} \mathrm d t\\
&= L[\cos(5t)](3) \\
&= \frac{s}{s^2 + 5^2}|_{s=3} \\
&= \frac{3}{3^2 + 25} \\
&= \frac{3}{34}
\end{align*}

\begin{example}[Webwork Assignment 3 Question 4]
Given that $0 \leq t \leq 3$, and $f(t+3) = f(t)$, consider the periodic function
$$f(t) = 2-e^{-3t}$$ Sketch and compute its Laplace Transform. 
\end{example}

We can first sketch the function in the first specified interval.We note that $f(0) = 1$ and at $t=3$, $f(t)$ approaches 2. This is repeated for each multiple of 3. 
However, by applying the shift formula, we obtain
\begin{align*}
\int_3^6 f(t) e^{-st}\mathrm d t &= \int_3^6f(t-3)e^{-s(t-3)-3s} \\
&= e^{-3s}\int_0^6f(u)e^{-su} \mathrm d u \\
\end{align*}
We note that we can therefore obtain an expression for the Laplace Tranform

\begin{align*}
L[f] &= \int_0^3f(t)e^{-st} \mathrm d t +e^{-3s} \int_0^3f(t)e^{-st} \mathrm d t + e^{-6s}\int_0^3f(t)e^{-st}\mathrm d t + ...\\
&= \sum_{n=0}^{\infty}\left(e^{-3s}\right)^n\int_0^3f(t)e^{-st}\mathrm d t \\
&= \frac{1}{1-e^{-3s}}\int_0^3\left(2-e^{-3t}\right)e^{-st} \mathrm d t \\
&= \frac{1}{1-e^{-3s}}\left(\frac{2}{s} - \frac{2}{s}e^{-3s} + \frac{1}{3+s}e^{-3s-9} - \frac{1}{3+s}\right)
\end{align*}

\subsection{The Inverse Laplace Transform}
Given the Laplace Transform of a function $$F(s) = L[f(t)](s)$$ the \textbf{inverse Laplace Transform} is given by $$f(t) = L^{-1}[F(s)](t)$$
\begin{example}
Find the inverse Laplace Transform of $$L^{-1}\left[\frac{s}{s^2+16}\right]$$ and $$L^{-1}\left[\frac{s+7}{s^2+6s+13}\right]$$
\end{example}
For the first, we note that it is simply $f(t) = \cos(4t)$. For the second, we first complete the square on the bottom to get $$\frac{s+7}{(s+3)^2 + 2^2}$$ Now, we need to separate this into a sum to get $$\frac{s+3}{(s+3)^2 + 2^2} +2 \frac{2}{(s+3)^2 + 2^2}$$ We can now express the original function in terms of $\cos$ and $\sin$ to obtain $$e^{-3t}\cos(2t) + 2e^{-3t}\sin(2t)$$

\begin{example}
Compute $$L^{-1}\left[\frac{s-2}{s^2-s-6}\right]$$
\end{example}
We can separate the denominator into $(s-3)(s+2)$. We then solve the partial fractions to find that $$\frac{s-2}{s^2-s-6} = \frac{1}{5(s-3)} + \frac{4}{5(s+2)}$$ We can thus write the inverse Laplace Transform as 
\begin{align*}
L^{-1}\left[\frac{s-2}{s^2-s-6}\right] &= \frac{1}{5}L^{-1}\left[\frac{1}{s-3}\right] + \frac{4}{5}L^{-1}\left[\frac{1}{s+2}\right] \\
&= \frac{1}{5}e^{3t} + \frac{4}{5}e^{-2t}
\end{align*}

\section{November 2, 2016}
\subsection{Inverse Laplace Transform Cont'd}

\begin{example}
Determine $$L^{-1}\left[e^{-as}F(s)\right]$$
\end{example}

We note that this is equal to $f(t-a)U_a(t)$, where $U_a(t) = Step(t-a) = h(t-a)$.

\begin{example}
Determine $$L^{-1}\left[\frac{1-e^{-2s}}{s^2}\right]$$
\end{example}

By linearity, we find that it is equal to 
\begin{align*}
 L^{-1}\left[\frac{1-e^{-2s}}{s^2}\right]&= L^{-1} \left[\frac{1}{s^2}\right] - L^{-1}\left[\frac{e^{-2s}}{s^2}\right] \\
&= t-(t-2)U_2(t)
\end{align*}

\begin{example}
Determine $$L^{-1}\left[e^{-5s}\frac{1}{(s+4)^3}\right]$$
\end{example}
We first note that $$L^{-1}\left[\frac{1}{(s+4)^3}\right] = \frac{a}{2}t^2e^{-4t}$$ Thus, we get
\begin{align*}
L^{-1}\left[e^{-5s}\frac{1}{(s+4)^3}\right] &= \frac{1}{2}U_5(t)e^{-4(t-5)}(t-5)^2
\end{align*}




\begin{example}
Determine $$L^{-1}\left[\frac{e^{-3s}(2s+4)}{s^2+25}\right]$$
\end{example}

We note that this is $$U_3(t)\left(2\cos(5(t-3)) + \frac{4}{5}\sin(5(t-3))\right)$$

\subsection{Differential Equations with the Laplace Transform}

To determine the solution, take the Laplace Transform of both sides where 

\begin{example}
Given that $t>0$, $y(0) = 1$ and $y'(0) = -1$, solve the initial value problem of $$y''(t) - y'(t) -6y = 0$$
\end{example}

We recall that the first step is to determine the Laplace transform of both sides. That is, we have $$L[y''] = s^2L[y] -sy(0) - y'(0) = s^2L[y]-s+1$$
$$L[y'] = sL[y] - y(0) = sL[y] - 1$$
$$L[y] = Y$$
Substituting, we get $s^2Y - s + 1 -sY + 1 -6Y = 0$. Rearranging this, we get $$Y = \frac{s-2}{s^2-s-6} = \frac{s-2}{(s-3)(s+2)}= \frac{\frac{1}{5}}{s-3} + \frac{\frac{4}{5}}{s+2}$$ Therefore, we can now apply the Inverse Laplace Transform on $Y$ to get the solution $$y(t) = \frac{1}{5}e^{3t}+\frac{4}{5}e^{-2t}$$

\begin{example}
Given that $y(0) = 2$ and $y'(0) = 1$, solve $$y'' + y = \sin(2t)$$
\end{example}

We first take the Laplace Transform of both sides to get $$L[y''] = s^2Y -sy(0) -y'(0) = s^2Y-2s-1$$ $$L[y] = Y$$
$$L\left[\sin(2t)\right] = \frac{2}{s^2+4}$$
Solving this, we find that $$Y = \frac{2s+1}{s^2+1} + \frac{2}{(s^2+1)(s^2+4)}$$
The second fraction can be separated to obtain $$Y = \frac{2s}{s^2+1} + \frac{1}{s^2+1} +  \frac{\frac{2}{3}}{s^2+1} + \frac{-\frac{2}{3}}{s^2+4} $$
We now note that
\begin{align*}
	y(t) &= 2L^{-1}\left[\frac{s}{s^2+1}\right] +\frac{5}{3}L^{-1}\left[\frac{1}{s^2+1}\right] -\frac{1}{3}L^{-1}\left[\frac{2}{s^2+4}\right]\\
	&= 2\cos(t) + \frac{5}{3}\sin(t) - \frac{1}{3}\sin(2t)
\end{align*}





\section{November 4, 2016}
\subsection{Problems of Inverse Laplace Transform}

\begin{remark}
$U_a(t) = Step(t-a) = h(t-a)$.
\end{remark}


\begin{example}
Given that $$f(t) = \begin{cases}
-8, & 0 \leq t \leq 8 \\
-1, & t \geq 8
\end{cases}$$ and $y(0) = 7$, solve $$y' + y = f(t)$$
\end{example}
We note that $f(t) = -8 + (-1-(-8))U_8(t) = -8+7U_8(t)$. Thus, we know that $L[f] = -\frac{8}{s}+\frac{7e^{-8s}}{s}$. Now, we can apply the Laplace Transform to get
\begin{align*}
L[y' + y] &= L[y'] + L[y] = sY -y(0) + Y = (s+1)Y+7\\
L[f] & = sY -y(0) + Y = (s+1)Y+7\\
(s+1)Y &= -7-\frac{8}{s}+\frac{7e^{-8s}}{s} \\
\end{align*}
That is, we have $$Y = -\frac{7s+8}{s(s+1)} + \frac{7}{s(s+1)}e^{-8s}$$ Separating the terms, we get that 
\begin{align*}
y &= L^{-1}\left[-\frac{8}{s}+\frac{1}{s+1} + \frac{7e^{-8s}}{s} - \frac{7e^{-8s}}{s+1}\right]\\
&= -8+e^{-t}+7U_8(t) -7e^{-(t-8)}U_8(t)\\
&= \begin{cases}e^{-t}-8, & 0 \leq t \leq 8 \\ 1+e^{-t}-e^{-t+8}, & t \geq 8\end{cases}
\end{align*}


\subsection{Systems of Linear Ordinary Differential Equations}
Suppose that we have $n$ unknown functions $y_1, y_2, ..., y_n$ of $t$ and a system of linear homogeneous equations. 
\begin{align*}(1)
y_1' &= a_{11}y_1 + ... + a_{1n}y_n \\
y_2' &= a_{21}y_1 + ... + a_{2n}y_n \\
...\\
y_n' &= a_{n1}y_1 + ... + a_{nn}y_n
\end{align*}
Any higher order equation can be written as a system. 

\begin{example}
The equation of a spring is given by $$y'' + \frac{k}{m}y = 0$$ or as $$y'' + \omega^2y = 0$$ It can be written as a system where $y_1 = y$ and $y_2 = y_1'$
\end{example}
The system is therefore given as 
$$y_1' = y_2 = 0y_1 + y_2 $$
$$y'' = y_2' = -\omega^2y_1 = -\omega^2y_1 + 0y_2$$

\begin{remark}
The general theory is the same for higher order equations.
\end{remark}

\begin{enumerate}
\item (Superposition Principle)

 If $Y_1 = \begin{pmatrix}y_1\\.\\.\\.\\y_n\end{pmatrix}$ and $Y_2 = \begin{pmatrix}z_1\\.\\.\\.\\z_n\end{pmatrix}$ are solutions of $(1)$, then $$a_1Y_1 + a_2Y_2$$ is also a solution. We note that $(1)$ can be written as $$Y' = AY$$ where $Y = \begin{vmatrix}y_1\\.\\.\\.\\y_n\end{vmatrix}$ and $A = \begin{vmatrix}a_{11} & .... & a_{1n}\\ a_{21} & ... & a_{2n} \\a_{n1} & ... & a_{nn}\end{vmatrix}$
 \item If we have $n$ linearly independent solutions $Y_1, ..., Y_n$, then $$Y = C_1Y_1 + ... + C_nY_n$$ where $C_i$ are constants is the general solution.
 \item We can define the Wronskian as $$W[Y_1, ..., Y_n] (t) =  \begin{vmatrix}
 	\begin{vmatrix}Y_1\end{vmatrix} & \begin{vmatrix}Y_2\end{vmatrix} & ... &\begin{vmatrix}Y_n\end{vmatrix}
 \end{vmatrix}$$ it is equal to zero if and only if $Y_1, ..., Y_n$ are linearly dependent (otherwise, $W \neq 0$ for any $t$).
\end{enumerate}











\section{November 7, 2016}
\subsection{Systems of Linear Differential Equations Cont'd}

Given that $x_1, x_2, ..., x_n$ are unknown functions, then
$$x_1' = a_{11}x_1 + a_{12}x_2, ..., a_{1n}x_n$$
$$x_2' = a_{21}x_1 + a_{22}x_2, ..., a_{2n}x_n$$
$$x_n' = a_{n1}x_1 + a_{n2}x_2, ..., a_{nn}x_n$$
This is a system of linear ordinary equations which can be written in the matrix form $$X' = AX$$ where $$X = \begin{bmatrix}x_1 \\x_2 \\ x_n\end{bmatrix}, A = \begin{bmatrix}a_{11} & a_{12} & a_{1n}\\a_{21} & a_{22} & a_{2n} \\ a_{n1} & a_{n2} & a_{nn}\end{bmatrix}$$

We note that the theory for homogeneous systems is the same for higher order equations. We have the following properties
\begin{enumerate}
\item If $X_1$ and $X_2$ are solutions, then $$X = c_1X_1 + c_2X_2$$ is a solution for any constants $c_1$ and $c_2$.
\item For any $n$ solutions $X_1, X_2, ...,X_n$, we define the Wronskian as $$W[X_1, X_2, ..,X_n](t) = \begin{vmatrix}\begin{bmatrix}X_1\end{bmatrix} & \begin{bmatrix}X_2\end{bmatrix} & ... & \begin{bmatrix}X_3\end{bmatrix}\end{vmatrix}$$
\item If solution $X_1, X_2, ..., X_n$ are linearly independent ($W(t) \neq 0$), then the general solution is $$X = C_1X_1 + C_2 X_2 + ... + C_nX_n$$
\end{enumerate}

Let us determine the method required to solve $X' = AX$. We first look for solutions in the form $X(t) = e^{\lambda t}\vec{v}$, where $\vec{v}$ is a constant vector. Then $X' = \lambda e^{\lambda t}\vec{v}$. Substituting this into the equation $X' = AX$, we obtain $$\lambda e^{\lambda t}\vec{v} = Ae^{\lambda t}\vec{v}$$ Solving this, we get $$\left( \lambda I-A\right)\vec{v} = 0$$ This is the characteristic equation. We note that $\lambda$ is an eigenvalue, and $A$ is an eigenvalue. This has a nontrivial solution $\vec{v}$ if and only if $$det(\lambda I-A) = 0$$ 

\begin{example}
Given that $x_1(0) = -3$ and $x_2(0) = -1$, solve the initial value problem for the following system of differential equations $$x_1' = x_1+2x_2$$ $$x_2' = 2x_1+4x_2$$
\end{example}

We recall that to determine the solution, we write system in matrix form where $$X = \begin{bmatrix}x_1 \\ x_2\end{bmatrix}, A = \begin{bmatrix}1 & 2 \\ 2 & 4\end{bmatrix}$$ We now determine the eigenvalues of $A$ to obtain 
\begin{align*}
det(A - \lambda I)& = \begin{vmatrix}1-\lambda & 2 \\ 2 & 4-\lambda \end{vmatrix} \\
&= (1-\lambda)(4-\lambda) -2*2\\
&= 4 - 4\lambda -\lambda + \lambda^2 - 4 \\ 
&= \lambda^2 -5\lambda \\
&= \lambda(\lambda-5) = 0
\end{align*}
Thus, we obtain $\lambda_1 = 0$ and $\lambda_5 = 5$. Now, we find the eigenvectors of $A$. For $\lambda_1 = 0$ and $\lambda_2 = 5$, we get $$A - 0I = \begin{bmatrix}1 & 2 \\2 & 4\end{bmatrix}A\vec{v} = 0$$ $$A-5I = \begin{bmatrix}1-5 & 2 \\ 2 & 4-5\end{bmatrix}$$ which gives eigenvectors of $$v_1 = \begin{bmatrix}-2 \\ 1\end{bmatrix}$$$$v_2 = \begin{bmatrix}1 \\ 2\end{bmatrix}$$ We now write the solutions where $$X_1 = e^{0t}\begin{bmatrix}-2 \\ 1\end{bmatrix} = \begin{bmatrix}-2 \\ 1\end{bmatrix}$$ $$X_2 = e^{5t}\begin{bmatrix}1 \\ 2\end{bmatrix} $$ The general solution is therefore 
\begin{align*}X &= C_1X_1 + C_2X_2\\
&=C_1\begin{bmatrix}-2 \\ 1\end{bmatrix} + C_2e^{5t}\begin{bmatrix}1 \\ 2\end{bmatrix}
\end{align*}
Now, given the initial conditions, we determine that at $t=0$, we have $C_1 = 1$ and $C_2 = -1$. Therefore, the solution to the initial value problem is therefore $$X(t) = \begin{bmatrix}-2 \\ 1\end{bmatrix} + e^{5t}\begin{bmatrix}-1 \\ -2\end{bmatrix}$$


\section{November 9, 2016}
\subsection{Systems of Linear Homogeneous Ordinary Differential Equations}

\begin{example}
Find the general solution of the system $$x_1' = 4x_1+x_2+x_3$$
$$x_2' = x_1 + 4x_2 + x_3$$
$$x_3' = x_1 + x_2 + 4x_3$$
\end{example}

We recall that we first rewrite the system as a matrix of the form $X' = AX$. That is, $$X = \begin{bmatrix}x_1\\x_2\\x_3\end{bmatrix}, A = \begin{bmatrix}4 & 1 & 1 \\ 1 & 4 & 1 \\ 1 & 1 & 4\end{bmatrix}$$
Now, we find the eigenvalues, such that 
\begin{align*}
|A-\lambda I| &= \begin{vmatrix}4-\lambda & 1 & 1 \\ 1 & 4-\lambda & 1 \\ 1 & 1 & 4-\lambda\end{vmatrix}\\
&= \begin{vmatrix}4-\lambda & 1 & 1 \\ -3+ \lambda & 3-\lambda & 0  \\-3+\lambda &0&3-\lambda\end{vmatrix}\\
&= (\lambda-3) \begin{vmatrix}4-\lambda & 1 & 1 \\ 1 & -1 & 0\\\lambda-3 & 0 & -(\lambda-3)\end{vmatrix}\\
&= (\lambda-3)^2\begin{vmatrix}4-\lambda & 1 & 1\\ 1 & -1 & 0\\ 1 & 0 & -1\end{vmatrix}\\
&=  (\lambda-3)^2\begin{vmatrix}6-\lambda & 0 & 0\\ 1 & -1 & 0\\ 1 & 0 & -1\end{vmatrix}\\
&= (\lambda-3)^2(6-\lambda)\begin{vmatrix}-1 & 0 \\0 & -1\end{vmatrix}\\
&= (\lambda-3)^2(6-\lambda)
\end{align*}
Thus, $\lambda_1=\lambda_2 = 3$ and $\lambda_3 = 6$. Now, we find eigenvectors for the eigenvalues. For $\lambda_1=\lambda_2 = 3$, we have $$A-3I = \begin{bmatrix}1 & 1 & 1\\ 1 & 1 & 1 \\ 1 & 1 & 1\end{bmatrix}$$ That is, $x+y+z=0$ with two free variables $y$ and $z$. Therefore $x = -y-z$ so taking $v_1: y = 1; z= 0$ and $v_2: y=0, z=1$, we have $$v_1=\begin{bmatrix}-1 \\ 1 \\ 0\end{bmatrix}, v_2=\begin{bmatrix}-1 \\ 0 \\ 1\end{bmatrix}$$
For $\lambda = 6$, we have
$$A - 6I = \begin{bmatrix}-2 & 1 & 1\\ 1 & -2 & 1 \\ 1 & 1 & -2\end{bmatrix} = \begin{bmatrix}1 & 1 & -2\\ 0 & 1 & -1 \\ 0 & 0 & 0\end{bmatrix}$$
That is, $x+y-2z=0$ and $y-z=0$. Therefore $x = y= z$ so $v_3:x = 1, y = 1, z= 1$ gives
$$v_3 = \begin{bmatrix}1 \\ 1 \\1\end{bmatrix}$$The general solution is therefore $$X = C_1e^{3t}\begin{bmatrix}-1 \\ 1\\ 0\end{bmatrix} + C_2e^{3t}\begin{bmatrix}-1 \\ 0\\ 1\end{bmatrix} + C_3e^{6t}\begin{bmatrix}1 \\ 1\\ 1\end{bmatrix}$$
Thus, this can also be expressed as 
$$x_1(t) =- C_1e^{3t}- C_2e^{3t}+ C_3e^{6t}$$
$$x_2(t) = C_1e^{3t}+ C_3e^{6t}$$
$$x_3(t) = C_2e^{3t}+ C_3e^{6t}$$



\begin{remark}
We note that eigenvalues are unique, but eigenvectors are not, as we can have infinitely many eigenvectors that correspond to the equation. 
\end{remark}
\begin{remark}If a real matrix has a complex eigenvalue $\lambda = a+bi$, then $\lambda_2 = \bar{\lambda_1} = a-bi$ is also an eigenvalue. If $v_1$ is an eigenvector associated with $\lambda_1$, then $\bar{v_1}$ is associated with $\lambda_2 = \bar{\lambda_1}$. If we have $\lambda$ complex and $v$ associated, the real and imaginary parts of $e^{\lambda t}$v are solutions. 
\end{remark}

\begin{example}
Solve the initial value problem given that $X(0) = \begin{bmatrix}3\\5\end{bmatrix}$ for the equation $$X' = \begin{bmatrix}0 & 1 \\ -1 & 0\end{bmatrix}X$$
\end{example}

We first find the eigenvalues. We note that 
\begin{align*}
|A-\lambda I| &= \begin{vmatrix}-\lambda & 1 \\ -1 & -\lambda\end{vmatrix}\\
&= \lambda^2 + 1 \\
\end{align*}
We note that $\lambda = \pm i$. Thus, to find the eigenvectors we have $$A-iI = \begin{bmatrix}-i & 1 \\-1 & -i\end{bmatrix}= \begin{bmatrix}-i & 1 \\0 & 0\end{bmatrix}$$ That is, $ix + y = 0$. Therefore $y = ix$ so $v:x=1, y = i$ gives $$v = \begin{bmatrix}1\\i\end{bmatrix}$$
The general solution is therefore 
\begin{align*}
X(t) &= e^{it}\begin{bmatrix}1 \\ i\end{bmatrix}\\
&= e^0(\cos(t) + i\sin(t))\begin{bmatrix}1 \\ i\end{bmatrix}\\
&= \begin{bmatrix}\cos(t) & i\sin(t) \\
i\cos(t) & i^2\sin(t)\end{bmatrix} \\
&= \begin{bmatrix}\cos(t) \\ -\sin(t) \end{bmatrix} + i \begin{bmatrix}\sin(t) \\ \cos(t) \end{bmatrix}\\
&= C_1\begin{bmatrix}\cos(t) \\ -\sin(t) \end{bmatrix} + C_2 \begin{bmatrix}\sin(t) \\ \cos(t) \end{bmatrix}
\end{align*}
Solving for $X(0)$ gives $C_1=3$ and  $C_2=5$. Thus the solution to the initial value problem is 
$$X(t) = 3\begin{bmatrix}\cos(t) \\ -\sin(t) \end{bmatrix} + 5 \begin{bmatrix}\sin(t) \\ \cos(t) \end{bmatrix}$$


\section{November 25, 2016}
\subsection{Sturn-Liouville Problems Cont'd}

\begin{example}
\end{example}

In the case that  $\lambda > 0$, the characteristic equation becomes $r^2 + \lambda = 0$. That is, $r = \pm \sqrt{\lambda}i$. The solution is therefore of the form $$X(x) = A\cos(\sqrt{\lambda}x) + B\sin(\sqrt{\lambda} x)$$
Substituting $X(0) = 0$ and $X(\pi) = 0$, we get that $A = 0$ and $X(\pi) = B\sin(\sqrt{\lambda}\pi)$, so for $B \neq 0$, $\sin(\sqrt{\lambda}\pi) = 0$. This means that $\sqrt{\lambda} =n$, so $\lambda_n = n^2$. Therefore, $$X_n(x) = \sin(nx)$$ The eigenvalues are $\lambda_n = n^2$, and the eigenfunctions are $X_n(x) = \sin(nx)$.

Generally, for $$X'' = \lambda X = 0$$ such that $X(0) = 0$ and $X(l) = 0$, the eigenvalues are $$\lambda_n = \left(\frac{n\pi}{l}\right)^2$$ and the eigenfunctions are $$X_n(x) = \sin\left(\frac{n\pi}{l}x\right)$$

When $X'(0) = 0$ and $X'(l) = 0$, where $\lambda_0 = 0$ and $X_0(x) = 1$, then we have 
$$\lambda_n = \left(\frac{n\pi}{l}\right)^2$$ and $$X_n(x) = \cos\left(\frac{n\pi}{l}x\right)$$


\begin{example}
Solve for $u(x,t)$ given as 
$$\frac{\partial u}{\partial t} = \frac{\partial^2 u}{\partial x^2}$$
where $u(0,t) = 0$, $u(\pi, t) = 0$, and $u(x,0) = \pi - x$ is the initial condition, and $t \geq 0$ and $0 < x < \pi$. 
\end{example}

We note that this is a Dirichlet Boundary Condition. We separate $x$ and $t$ assuming that $u(x,t) = X(x) T(t)$.
First, we now note that $\frac{\partial u}{\partial t} = XT'$ and $\frac{\partial^2 u}{\partial x^2} = X''T$. Now, we have $$XT' = X''T$$ We can now separate this and write it as $$\frac{T'}{T} = \frac{X''}{X} = -\lambda$$ where the first term depends on $t$, the second depends on $x$, and $\lambda$ is a constant. We now have $$\frac{X''}{X} = -\lambda \rightarrow X'' + \lambda X = 0$$
Now, we find the boundary conditions, eigenvalues and eigenfunctions of the Sturn-Liouville problem. Now, $$u(0,t) = X(0)T(t) = 0 \rightarrow X(0) = 0$$
$$u(\pi, t) = X(\pi) T(t) = 0 \rightarrow X(\pi) = 0$$
The eigenvalues are $\lambda_n = n^2$ and the eigenfunctions are $X_n(x) = \sin(nx)$.

Now, we have $$\frac{T_n'}{T_n} = -\lambda_n = -n^2 \rightarrow T_n' = -n^2T_n$$ That is, we have 
$$T_n(t) = e^{-n^2t} \rightarrow u_n(x,t) = e^{-n^2t}\sin(nx)$$ where the first term is $T_n(t)$ and the second term is $X_n(x)$. Now, applying the superposition principle, a combination is a solution, so $$u(x,t) = \sum_{n=1}^{\infty}b_ne^{-n^2t}\sin(nx)$$
We now satisfy the initial condition using the Fourier series. Since $u(x,0) = \pi -x$, this means that 
\begin{align*}
u(x,0) &= \sum_{n=1}^{\infty}b_ne^0\sin(nx) \\
&= \sum_{n=1}^{\infty}\left(\frac{2}{\pi}\int_0^{\pi}(\pi - x)\sin(nx) \mathrm d x\right)\sin(nx) \\
&= \sum_{n=1}^{\infty}\left(\frac{2}{\pi}\left[\frac{x - \pi}{n}\cos(nx) - \frac{\sin(nx)}{n^2} + C\right]\Big|_0^{\pi}\right)\sin(nx) \\
&= \sum_{n=1}^{\infty}\left(\frac{2}{\pi}*\frac{\pi}{n}\right)\sin(nx) \\
&= \sum_{n=1}^{\infty}\left(\frac{2}{n}\right)\sin(nx) \\
\end{align*}

Thus, now that we know $b_n = \frac{2}{\pi}$, then
$$u(x,t) = \sum_{n=1}^{\infty} \frac{2}{n}e^{-n^2t}\sin(nx)$$

\section{Separation of Variables Cont'd}

\begin{example}
Solve the initial boundary value problem of the heat conduction equation given that $0 < x < \pi$ and $t > 0$ for the equation $$5\frac{\partial ^2 u}{\partial x^2} = \frac{\partial u}{\partial t}$$ with $\frac{\partial u}{\partial x}\Big|_{x=0}=0$, $\frac{\partial u}{\partial x}\Big|_{x=\pi}=0$ and $u(x,0) = \cos(3x)$.
\end{example}

First, we assume that $u(x,t) = X(x)T(t)$ to obtain the Sturn-Liouville problem for $X(x)$. That is, we obtain $$\frac{\partial^2 u}{\partial x^2} = T(t)X''(x)$$
$$\frac{\partial u}{\partial t} = X(x)T'(t)$$
Thus, the original equation becomes $$5TX'' = XT'$$ We now separate the terms to obtain $$\frac{X''}{X} = \frac{T'}{5T} = -\lambda$$ 
where $\lambda$ is a constant. Now, $\frac{X''}{X} = -\lambda$. Furthermoe, we know that $\frac{\partial u}{\partial x} = X'T$. By the initial conditions, we know that 
$$\frac{\partial u}{\partial x}\Big|_{x=0} = X'(0)T(t) = 0 \rightarrow X'(0) = 0$$
$$\frac{\partial u}{\partial x}\Big|_{x=\pi} = X'(\pi)T(t) = 0 \rightarrow X'(\pi) = 0$$
The Sturn-Liouville problem for $X$ is therefore $$\frac{X''}{X} = -\lambda$$ or $$X'' + \lambda X = 0$$ with $X'(0) = 0$ and $X'(\pi) = 0$. 
Secondly, we solve the Sturn-Liouville problem. For $\lambda < 0$, we find that the characteristic equation gives $$r^2 + \lambda = 0$$ implying that $r_{1,2} = \pm\sqrt{-\lambda} = \pm k$ where $k =\sqrt{-\lambda} > 0$. Thus, we have 
$$X(x) = A\cosh(kx) + B\sinh(kx)$$
$$X'(x) = Ak\sinh(kx) + Bk\cosh(kx)$$
Solving for the initial conditions, we find that $$X'(0) = Ak\sinh(0) + Bk\cosh(0) = 0 \rightarrow B = 0$$
$$X'(\pi) = Ak\sinh(k\pi) = 0 \rightarrow A = 0$$
So $\lambda<0$ is not an eigenvalue. In the second case where $\lambda = 0$, we find that the characteristic equation becomes $$r^2 = 0$$ thus implying that $r_{1,2} = 0$. Thus, we have 
$$X(x) = Ax + B$$ 
$$X'(x) = A$$
Solving for the initial conditions, we find that  
$$X'(0) = A = 0$$
$$X'(\pi) = A = 0$$
This implies that $B$ can be any value. Therefore, $\lambda_0=0$ is an eigenvalue and $X_0 = 1$ is an eigenfunction. In the third case that $\lambda>0$, the characteristic equation is $$r^2 + \lambda = 0$$, implying that $r = \pm\sqrt{\lambda}i$. Therefore, we have $$X(x) = A\cos(\sqrt{\lambda}x) + B\sin(\sqrt{\lambda}x)$$
$$X'(x) = -A\sqrt{\lambda}\sin(\sqrt{\lambda}x) + B\sqrt{\lambda}\cos(\sqrt{\lambda}x)$$
Solving for the initial conditions gives $B = 0$ and $$X'(\pi) = -A\sqrt{\lambda}\sin(\sqrt{\lambda}\pi) = 0 \rightarrow \sqrt{\lambda} = n$$ since $A \neq 0$, so $\lambda_n = n^2$ and so $X_n(x) = \cos(nx)$.
The third step involves the use of superposition. That is, $$\frac{1}{5}\frac{T'}{T} = -\lambda_n$$, and so $$T' = -5\lambda_nT$$
This implies that $\lambda_0 = 0$ and $\lambda_n = n^2$

In the fourth step, we find the coefficients using the initial conditions (Fourier sine or cosine series ), where sine is for dirichlet and cosine is for neumann.

\section{November 30, 2016}
\subsection{}
\begin{example}
$$\begin{cases}5\frac{\partial^2u}{\partial x^2} = \frac{\partial u}{\partial t}, 0 < x < \pi, t > 0\\
\frac{\partial u}{\partial x}\Big|_{x=0} = 0, \frac{\partial u}{\partial x}\Big|_{x=\pi} = 0, t > 0\\
u(x,0) = \cos(3x), 0 < x < \pi\end{cases}$$
\end{example}

The fourth step involves solving for the initial conditions. That is, we have solve for
$$u(x,t) = a_0 + \sum_{n=1}^{\infty}a_n\cos(nx)e^{-5n^2t}$$ using $t=0$. We know that at $t=0$, $u(x,0) = \cos(3x)$. We now employ the cosine series for $\cos(3x)$. Therefore, we find that $$a_0 = 0$$ $$a_n = \begin{cases}0, n \neq 3 \\ 1, n = 3\end{cases}$$ Therefore, the solution is $$u(x,t) = \cos(3x)e^{-45t}$$

\begin{example}
Given that $u(0,t) = 0$, $u(\pi, t) = 0$ for $t>0$ and $u(x,0) = 0$ with $\frac{\partial u}{\partial t}(x,0) = 12\sin(3x)$ for $ 0 < x < \pi$, solve the initial boundary value problem of $$4\frac{\partial^2 u}{\partial x^2} = \frac{\partial^2 u}{\partial t^2}, t > 0, 0 < x < \pi$$
\end{example}

We write $u(x,t) = X(x)T(t)$. Thus, the equation becomes $$4X''T = XT''$$ Rearranging this equation, we obtain $$\frac{X''}{X} = \frac{T''}{4T} = -\lambda$$
That is, $$X'' + \lambda X = 0$$ $$T'' + 4\lambda T = 0$$
Now, we note that $u(0,t) = X(0) T(t) = 0$ and since $T(t) \neq 0$, then this implies that $X(0) = 0$. Similarly, $u(\pi, t) = X(\pi)T(t) = 0$, which implies that $X(\pi) = 0$. 
Secondly, we now solve the Sturn-Liouville problem where 
$$X'' + \lambda X = 0$$ where $X(0) = 0$ and $X(\pi) = 0$. We could now consider the cases where $\lambda < 0$, $\lambda = 0$ and $\lambda > 0$. However, we simply use the known answer to obtain $$\lambda_n = n^2\left(\frac{\pi n}{l}\right)^2$$ in the case that $0 < x < l$, where $X_n = \sin(nx)$.
Thirdly, we compute $T_n(t)$ by noting that $T_n'' = -4\lambda_nT_n = 0$, which can be rewritten as $$T_n'' = -4n^2T_n = 0$$ Finding the characteristic equation to get $r^2 + 4n^2 = 0$, we find that $r = \pm\sqrt{4n^2} = \pm 2ni$. That is, 
$$T_n(t) = A_n\cos(2nt) + B_n\sin(2nt)$$ Now by using the superposition principle for both $T_n(t) $ and $X_n(x)$, we get 
$$u(x,t) = \sum_{n=1}^{\infty}\left[ A_n\cos(2nt) + B_n\sin(2nt)\right]\sin(nx)$$
Lastly, we solve for the initial conditions. Thus, we find that 
$$u(x,0) = \sum_{n=1}^{\infty}A_n\sin(nx) = 0$$
for $0 < x < \pi$. That is, $A_n = 0$ for all $n = 1, 2, 3, ...$ Thus, 
$$u(x,t) = \sum_{n=1}^{\infty}\left[ B_n\sin(2nt)\right]\sin(nx)$$
Solving for the other initial condition, we have 
\begin{align*}
\frac{\partial u}{\partial t} \Big|_{t=0} &= \sum_{n=1}^{\infty}2nB_n\cos(2nt)\sin(nx)\\
&= \sum_{n=1}^{\infty}2nB_n\sin(nx) = 12\sin(3x)
\end{align*}
This implies that 
$$B_n =\begin{cases}0, n \neq 3\\2, n = 3 \end{cases}$$Therefore, the solution is 
$$u(x,t) = 2\sin(6t)\sin(3x)$$

\begin{example}
Determine for which $c$ the function $u(x,t) = \cos(5x)\sin(25t)$ is a solution of the wave equation 
$$c\frac{\partial ^2 u }{\partial x^2} = \frac{\partial ^2 u}{\partial t^2}$$ Additionally, determine the boundary conditions given that $0 < x < \pi ,t > 0$.
\end{example}

We note that $u_{xx} = -25u$ and $u_{tt} = -25^2u$. Substituting these values into the equation, we get 
$$c(-25u) = -25^2u$$ This implies that $c = 25$. To determine the boundary conditions, we note that 
$$\frac{\partial u}{\partial x}\Big|_{x=0} = 0$$
$$\frac{\partial u}{\partial x}\Big|_{x=\pi} = 0$$



\section{Paragraph}
In \LaTeX, paragraphs are causefewf fsdfsdf sd sf s sdf 
when two line breaks are used.
Single line breaks are ignored.
Hence this entire block is one paragraph.

Now this is a new paragraph. If you want to
start a new line without a new paragraph, use
two backslashes like this:
\\
Now the next words will be on a new line.
\textbf{As a general rule, use this as infrequently as possible.}

You can \textbf{bold} or \textit{italicize} text.
Try to not do so repeatedly for mechanical tasks by, e.g. using theorem environments (see Section \ref{sec:theorem}).


\section{Math}
Inline math is created with dollar signs,
like $e^{i \pi} = -1$ or $\half \cdot 2 = 1$.

Display math is created as follows:
\[ \sum_{k=1}^n k^3 = \left( \sum_{k=1}^n k \right)^2. \]
This puts the math on a new line. Remember to properly add punctuation to the end of your sentences -- display math is considered part of the sentence too!

Note that the use of \verb \left(  causes the parentheses to be the correct size. Without them, get something ugly like
\[ \sum_{k=1}^n k^3 = ( \sum_{k=1}^n k )^2. \]

\subsection{Using alignment}
Try this:
\begin{align*}
	\prod_{k=1}^4 \left( i-x_k \right)\left( i+x_k \right) &= P(i) \cdot P(-i) \\
	&= \left( 1-b+d+i(c-a) \right)\left( 1-b+d-i(c-a) \right) \\
	&= (a-c)^2 + \left( b-d-1 \right)^2. 
\end{align*}

\section{Shortcuts}
In the beginning of the document we wrote
\begin{verbatim}
\newcommand{\half}{\frac{1}{2}}
\newcommand{\cbrt}[1]{\sqrt[3]{#1}}
\end{verbatim}
Now we can use these shortcuts.
\[ \half + \half = 1 \text{ and } \cbrt{8} = 2. \]

\section{Theorems and Proofs}
\label{sec:theorem}
% ^ Now we can refer to this
Let us use the theorem environments we had in the beginning.
\begin{definition}
	Let $\mathbb R$ denote the set of real numbers.
\end{definition}
Notice how this makes the source code READABLE.

\begin{theorem}
	[Vasc's Inequality]
	\label{thm:vasc}
	For any $a$, $b$, $c$ we have the inequality
	\[ \left( a^2+b^2+c^2 \right)^2 \ge 3\left( a^3b+b^3c+c^3a \right). \]
\end{theorem}

For the proof of Theorem \ref{thm:vasc}, we need the following lemma.

\begin{lemma}
	We have $\left( x+y+z \right)^2 \ge 3(xy+yz+zx)$ for any $x,y,z \in \mathbb R$.
\end{lemma}
\begin{proof}
	This can be rewritten as
	\[ \half\left( (x-y)^2+(y-z)^2+(z-x)^2 \right) \ge 0 \]
	which is obvious.
\end{proof}

\begin{proof}
	[Proof of Theorem \ref{thm:vasc}]
	In the lemma, put $x=a^2-ab+bc$, $y=b^2-bc+ca$, $z=c^2-ca+ab$.
\end{proof}

\begin{remark}
	In \autoref{thm:vasc}, equality holds if $a : b : c = \cos^2 \frac{2\pi}{7} : \cos^2 \frac{4\pi}{7} : \cos^2 \frac{6\pi}{7}$.
	This unusual equality case makes the theorem difficult to prove.
\end{remark}


\section{Referencing}
The above examples are the simplest cases.
You can get much fancier: check out
\href{http://en.wikibooks.org/wiki/LaTeX/Labels_and_Cross-referencing}{the Wikibooks}.

\section{Numbered and Bulleted Lists}
Here is a numbered list.
\begin{enumerate}
	\item The environment name is ``enumerate''.
	\item You can nest enumerates.
		\begin{enumerate}
			\item Subitem
			\item Another subitem
		\end{enumerate}
	\item[$2 \half$.] You can also customize any particular label.
	\item But the labels continue onwards afterwards.
\end{enumerate}

\bigskip

You can also create a bulleted list.
\begin{itemize}
	\item The syntax is the same as ``enumerate''.
	\item However, we use ``itemize'' instead.
\end{itemize}


\end{document}
